\documentclass[letterpaper, 12pt, oneside]{tesis}
\graphicspath{{figuras/}}

\usepackage[spanish]{babel}
\usepackage[utf8]{inputenc}
\usepackage[fixlanguage]{babelbib}
\usepackage{hyphenat}

\usepackage{natbib}
\usepackage{verbatim}
\usepackage{tikz}
\usepackage{fancyvrb}
\usepackage{enumerate}
\usepackage{paquetes/vector}
\usepackage{color}
\usepackage{hyperref}
\usepackage{amsmath}
\usepackage{amsthm}
\usepackage{amssymb}
\usepackage{amsfonts}
\usepackage{algorithmic}
\usepackage{algorithm}
\usepackage{float}
\usepackage{epigraph}
\usepackage{array}
\usepackage{tabularx}
\usepackage{caption}
\usepackage{multirow}
\usepackage{lscape}
%:%s/\(bibitem.\+\)and/\1y/g
%\usepackage[letterpaper, inner=3cm, outer=2cm]{geometry}
%\addtolength[\topmargin]{-1.5cm}
%\addtolength[\textheight]{4cm}

%\usepackage{setspace,cite} % Doble espacio para texto, espacio singular para
                           % los caption y pie de pagina

\hypersetup{urlcolor=blue, colorlinks=false}
\numberwithin{algorithm}{chapter}

\newcommand{\ALG}{Lista de Algoritmos}
\renewcommand*\listalgorithmname{\ALG}
\renewcommand{\tablename}{Tabla}
\renewcommand{\spanishtablename}{Tabla}
\renewcommand*\listtablename{Lista de Tablas}
\newcommand\listsymbolname{Acrónimos}
\newcommand{\tup}[1]{\langle #1 \rangle}
\floatname{algorithm}{Algoritmo}
\newcommand{\A}{\ensuremath{\mathcal{A}}}
\newcommand{\LPO}{\text{LPO}}
\newcommand{\LSO}{\text{LSO}}
\newcommand{\SO}{\ensuremath{\text{SO}}}
\newcommand{\SOE}{\ensuremath{\text{SO}\exists}}
\newcommand{\SOA}{\ensuremath{\text{SO}\forall}}
\newcommand{\SOEA}{\ensuremath{\text{SO}\exists\forall}}
\newcommand{\SOAEA}{\ensuremath{\text{SO}\forall\exists\forall}}
\newcommand{\Q}{\ensuremath{\mathcal{Q}}}
\newcommand{\struc}{\text{STRUC}}
\newcommand{\FT}{\ensuremath{\mathfrak{F}}}
\newcommand{\eq}{\equiv}
\renewcommand{\models}{\vDash}
\newcommand{\domred}{\mathfrak{D}}
\newcommand{\insred}{\mathfrak{I}}
\newcommand{\domain}{\mathsf{dom}}
\newcommand{\instance}{\mathsf{ins}}
\newcommand{\ground}{\mathfrak{G}}
\newcommand{\fttt}[1]{{\texttt{\footnotesize #1}}}
\newcommand{\prove}{{\texttt{\footnotesize prove}}}
\renewcommand{\t}{\ensuremath{\tau}}

\parindent 3ex % Agrega sangrías de 3 espacios (3 veces el espacio de la x)
\setlength{\baselineskip}{1.5pt} % Interlineado 1.5
\setlength{\parskip}{16.5pt} % Interparrafo 16.5pt
%\oddsidemargin 3cm
\topmargin 2cm

% Grammar
\newcommand{\deriv}{\longrightarrow}
% non-terminals
\newcommand{\sofbf}{\ensuremath{\langle\textsf{fbf-so}\rangle}}
\newcommand{\listrel}{\ensuremath{\langle\textsf{lista-rel}\rangle}}
\newcommand{\pofbf}{\ensuremath{\langle\textsf{fbf-po}\rangle}}
\newcommand{\listvar}{\ensuremath{\langle\textsf{lista-var}\rangle}}
\newcommand{\qffbf}{\ensuremath{\langle\textsf{qf-fbf}\rangle}}
\newcommand{\listpofbf}{\ensuremath{\langle\textsf{lista-fbf-po}\rangle}}
\newcommand{\atom}{\ensuremath{\langle\textsf{atom}\rangle}}
\newcommand{\tipof}{\ensuremath{\langle\textsf{tipo-fun}\rangle}}

% terminals
\newcommand{\lpar}{\texttt{(}}
\newcommand{\rpar}{\texttt{)}}
\newcommand{\AND}{\texttt{and}}
\newcommand{\OR}{\texttt{or}}
\newcommand{\NOT}{\texttt{not}}
\newcommand{\IMPLIES}{\texttt{implies}}
\newcommand{\SOExists}{\texttt{so-exists}}
\newcommand{\SOForall}{\texttt{so-forall}}
\newcommand{\Exists}{\texttt{exists}}
\newcommand{\Forall}{\texttt{forall}}
\newcommand{\Inj}{\text{Inj}}
\newcommand{\Fun}{\text{Fun}}
\newcommand{\PFun}{\text{PFun}}
\newcommand{\PInj}{\text{PFun}}

% tokens
\newcommand{\var}{\texttt{<VAR>}}
\newcommand{\const}{\texttt{<CONST>}}
\newcommand{\rel}{\texttt{<REL>}}
\newcommand{\integer}{\texttt{<int>}}
\newcommand{\fsymbol}{\texttt{<FUNC>}}

% holds
\newcommand{\holds}[2]{\ensuremath{\textbf{holds}[#1](#2)}}
\newcommand{\action}[4]{\ensuremath{\textbf{set}[#1](#2): \textbf{pre}=\{#3\},\ \textbf{add}=\{#4\}}}

% translation
\newcommand{\F}{\mathbb{F}}
\newcommand{\Not}{\text{not-}}
%\newcommand{\est}[4]{\textbf{est}[#1](#2): \textbf{pre}=\{#3\},\ \textbf{add}=\{#4\}}
\newcommand{\est}[4]{\textbf{establish}[#1](#2)=(\{#3\},\{#4\})}
\newcommand{\settrue}[3]{\textbf{set-true}[#1](#2)=(\emptyset,\{#3\})}
\newcommand{\setfalse}[3]{\textbf{set-false}[#1](#2)=(\emptyset,\{#3\})}
\newcommand{\SUC}{\text{SUC}}
\newcommand{\len}{\ell}

%\newcommand{\action}[1]{action[#1]}
\newcommand{\pre}{\text{pre}}
\newcommand{\eff}{\text{eff}}
\newcommand{\true}{\textbf{true}}
\newcommand{\false}{\textbf{false}}
\newcommand{\SAT}{\textsc{Sat}}
\newcommand{\CLIQUE}{\textsc{Clique}}
\newcommand{\CHD}{\textsc{CHD}}
\newcommand{\TDM}{\textsc{3DM}}
\newcommand{\TCOL}{\textsc{3Col}}
\newcommand{\KCOL}{\ensuremath{k\textsc{Col}}}
\newcommand{\qAE}{\ensuremath{\forall\exists}\xspace}
\newcommand{\qEA}{\ensuremath{\exists\forall}\xspace}
\newcommand{\qEAE}{\ensuremath{\exists\forall\exists}\xspace}
\newcommand{\qAEA}{\ensuremath{\forall\exists\forall}\xspace}
\newcommand{\coCOL}{\ensuremath{\overline{\text{3Col}}}\xspace}
\newcommand{\STRIPS}{\text{STRIPS}\xspace}
\newcommand{\TC}{\text{TC}}
\newcommand{\PH}{\ensuremath{\text{PH}}\xspace}
\newcommand{\PSPACE}{\text{PSPACE}\xspace}
\newcommand{\UNSAT}{\textsc{Unsat}}

\renewcommand{\O}{\ensuremath{\mathcal{O}}}
\newcommand{\pwin}{long}


%! referencias manera estandar

%%%% Título
\begin{titlepage}
	\title{\vspace{-2cm} \includegraphics[width=1.2in]{./usb.png} \\[.2cm]
		\large Universidad Simón Bolívar \\
		Decanato de Estudios Profesionales \\
		Coordinación de Ingeniería de la Computación
		\vfill
		\LARGE Reducciones automáticas de problemas de decisión en problemas de
        planificación \vfill}
	\author{Por: \\
		Aldo Fabrizio Porco Rametta \\
		Alejandro Machado González \\[1.2cm]
		Realizado con la asesoría de: \\
		Prof. Blai Bonet\\[1.2cm]
		PROYECTO DE GRADO \\
Presentado ante la Ilustre Universidad Simón Bolívar \\
como requisito parcial para optar al título de \\
Ingeniero de Computación}
    \date{Sartenejas, Noviembre de 2012}
\end{titlepage}
%%%% Título

\begin{document}

\frontmatter

\maketitle
\setstretch{1.3}

%% Define the page headers using the FancyHdr package and set up for one-sided printing
%\fancyhead{}  % Clears all page headers and footers
%\rhead{\thepage}  % Sets the right side header to show the page number
%\lhead{}  % Clears the left side page header
%
%\pagestyle{fancy}  % Finally, use the "fancy" page style to implement the FancyHdr headers
%
%%% ----------------------------------------------------------------
%% The "Funny Quote Page"
%\setcounter{page}{2}
%\pagestyle{empty}  % No headers or footers for the following pages
%
%\null\vfill
%% Now comes the "Funny Quote", written in italics
%\textit{``El hombre razonable se adapta al mundo; el irrazonable intenta adaptar
%el mundo a si mismo. Así pues, el progreso depende del hombre irrazonable.''}
%
%\begin{flushright}
%George Bernard Shaw
%\end{flushright}
%
%\vfill\vfill\vfill\vfill\vfill\vfill\null
%\clearpage  % Funny Quote page ended, start a new page
%%% ----------------------------------------------------------------
%
%%% ----------------------------------------------------------------

%The Abstract Page

\addtotoc{Resumen}  % Add the "Abstract" page entry to the Contents
\abstract{
\addtocontents{toc}{\vspace{1em}}  % Add a gap in the Contents, for aesthetics
Aquí va el resumen.

\noindent \textbf{Palabras clave}: Planificación Automática, Teoría de
Complejidad, Lógica, Ingeniería del Conocimiento, Inteligencia Artificial.

\clearpage % Abstract ended, start a new page
%% ----------------------------------------------------------------

\setstretch{1.3}  % Reset the line-spacing to 1.3 for body text (if it has changed)

% The Acknowledgements page, for thanking everyone
%\acknowledgements{
%\addtocontents{toc}{\vspace{1em}}  % Add a gap in the Contents, for aesthetics
%Agradecimientos aquí.
%}

%\clearpage  % End of the Acknowledgements

\pagestyle{fancy}  %The page style headers have been "empty" all this time, now use the "fancy" headers as defined before to bring them back

%% ----------------------------------------------------------------
\lhead{\emph{Índice General}}  % Set the left side page header to "Contents"
\tableofcontents  % Write out the Table of Contents

%% ----------------------------------------------------------------
\lhead{\emph{Índice de Figuras}}  % Set the left side page header to "List if figuras" 
\listoffigures  % Write out the List of figuras

%% ----------------------------------------------------------------
\lhead{\emph{Índice de Tablas}}  % Set the left side page header to "List of Tables"
\renewcommand*\listtablename{Lista de Tablas}
\listoftables  % Write out the List of Tables

%% ----------------------------------------------------------------
%\lhead{\emph{Lista de Algoritmos}} % Set the left side page header to "List of Algorithms"
%\addtotoc{Lista de Algoritmos}
%\listofalgorithms % Write out the List of Algorithms

%% ----------------------------------------------------------------
\setstretch{1.5}  % Set the line spacing to 1.5, this makes the following tables easier to read
\clearpage  % Start a new page
\lhead{\emph{Acrónimos y símbolos}}  % Set the left side page header to "Abbreviations"
\listofsymbols{ll}  % Include a list of Abbreviations (a table of two columns)
{
% \textbf{Acronym} & \textbf{W}hat (it) \textbf{S}tands \textbf{F}or \\
\textbf{3Col} & \textbf{3}-\textbf{Col}orabilidad\\
\textbf{3DM} & \textbf{3}-\textbf{D}imensional \textbf{M}atching\\
\textbf{CHD} & \textbf{C}amino \textbf{H}amiltoniano \textbf{D}irigido\\
\textbf{BNF} & \textbf{B}ackus-\textbf{N}aur \textbf{F}orm\\
\textbf{CNF} & \textbf{C}onjunctive \textbf{N}ormal \textbf{F}orm\\
\textbf{kCol} & $k$-\textbf{Col}orabilidad\\
\textbf{LPO} & \textbf{L}ógica de \textbf{P}rimer \textbf{O}rden\\
\textbf{LSO} & \textbf{L}ógica de \textbf{S}egundo \textbf{O}rden\\
\textbf{LSO} & \textbf{T}eoría de \textbf{C}omplejidad \textbf{D}escriptiva\\
\textbf{NP} & \textbf{N}ondeterministic \textbf{P}olynomial time (clase de complejidad)\\
\textbf{PDDL} & \textbf{P}lanning \textbf{D}omain \textbf{D}efinition \textbf{L}anguage \\
\textbf{PH} & \textbf{P}olynomial \textbf{H}ierarchy (clase de complejidad)\\
\textbf{QBF} & \textbf{Q}uantified \textbf{B}oolean \textbf{F}ormula\\
\textbf{SAT} & boolean \textbf{SAT}isfiability\\
\textbf{STRIPS} & \textbf{S}tanford \textbf{R}esearch \textbf{I}nstitute \textbf{P}roblem \textbf{S}olver\\
\textbf{TCD} & \textbf{T}eoría de \textbf{C}omplejidad \textbf{D}escriptiva\\
%&\\
%\hline
%&\\
%\textbf{Fun} & abreviación de función total\\
%\textbf{Inj} & abreviación de función total inyectiva\\
%\textbf{PFun} & abreviación de función parcial\\
%\textbf{PInj} & abreviación de función parcial inyectiva\\
&\\
\hline
&\\
$\iff$ & doble implicación, si y sólo si\\
$\Rightarrow$ & implicación lógica\\
$[u:=v]$ & sustitución textual de $u$ por $v$
}

%% ----------------------------------------------------------------
% End of the pre-able, contents and lists of things
% Begin the Dedication page

\setstretch{1.3}  % Return the line spacing back to 1.3

\pagestyle{empty}  % Page style needs to be empty for this page
%\dedicatory{Dedicated to adventures...\ldots}

\addtocontents{toc}{\vspace{2em}}  % Add a gap in the Contents, for aesthetics

%% ----------------------------------------------------------------
\mainmatter	  % Begin normal, numeric (1,2,3...) page numbering
\pagestyle{fancy}  % Return the page headers back to the "fancy" style

% Include the chapters of the thesis, as separate files
% Just uncomment the lines as you write the chapters

% Chapter 1

\chapter*{Introducción} % Write in your own chapter title
\label{Intro}
\lhead{\emph{Introducción}} % Write in your own chapter title to set the page header
\addcontentsline{toc}{chapter}{Introducción}

Introducción.

\cite{cita}.

% Chapter 2

\chapter{Marco Teórico}
\label{Chapter1}
\lhead{Capítulo 1. \emph{Marco Teórico}}

\section{Planificación automática}
\subsection{STRIPS}
\subsection{Complejidad computacional de subconjuntos de STRIPS}

\section{Complejidad Descriptiva}
\subsection{Lenguajes}
\subsection{Estructuras de primer orden}
\subsubsection{Orden y aritmética}
\subsection{Semántica formal}
\subsection{Clases de complejidad}
%\subsubsection{Equivalencias con complejidad en planificación automática}

\section{Trabajos previos}

% Chapter 2

\chapter{Traducción de problemas NP}
\label{Chapter2}
\lhead{Capítulo 2. \emph{Traducción de problemas NP}}
Este capítulo describe el diseño de una herramienta capaz de transformar una
descripción de un problema en lógica de segundo orden existencial
en un problema de planificación perteneciente a la clase de complejidad
equivalente (NP).
Primero se explica el funcionamiento a alto nivel de la herramienta, en función
de sus entradas y salidas. Luego, se define formalmente una función de
traducción de lógica a planificación, demostrando las propiedades pertinentes.
Finalmente, se presenta la gramática de un lenguaje de especificación basado en lógica 
y se describe brevemente la implementación de la herramienta, que consiste en un 
analizador sintáctico y función de traducción ya descrita.

\section{Perspectiva general}
Desde el punto de vista del usuario, la contribución de esta herramienta es
proveer una forma alternativa y descriptiva de especificar problemas de forma
que sea posible obtener e interpretar su solución.

Como puede verse en la Figura \ref{esquema_herramienta}, la herramienta tiene
como entradas una sentencia $\Phi\in\SOE(\sigma)$, y
una estructura finita de primer orden $\A\in\struc[\sigma]$ sobre la cual se
intentará probar la propiedad definida por $\Phi$.
Las salidas son un dominio y una instancia PDDL que tienen un plan válido si y
sólo si $\A$ satisface $\Phi$. Además, un \textbf{certificado} de la validez de
la propiedad en $\Phi$ puede obtenerse del plan en tiempo lineal.

\begin{figure}[h]
\centering
\includegraphics[width=\textwidth]{figuras/esquema_herramienta.pdf}
\caption{Esquema de operación de la herramienta}
\label{esquema_herramienta}
\end{figure}

La traducción se basa en convertir la tarea de demostrar la validez de la
fórmula lógica en una tarea de planificación. Por lo tanto, las acciones del
dominio de planificación harán las veces de las operaciones de semántica
formal, probando cada tipo de subfórmula según un esquema similar a la
definición \ref{semantica_def}. La información específica al problema, como el
estado inicial, está codificada por la estructura de primer orden que recibe la
herramienta. Se ha visto que estas estructuras pueden representar grafos, por lo
que el \textit{software} es especialmente útil para la modelación y resolución
de problemas de grafos.

\section{Reducción de LSO a STRIPS}
Una \textbf{reducción} de un problema $A$ a un problema $B$ es una función
computable $f$ tal que para cada instancia
$\omega$, $\omega\in A$ si y sólo si $f(\omega)\in B$.
La herramienta puede ser considerada un generador de reducciones entre problemas.

En este caso, la reducción se descompone en dos funciones
\begin{alignat*}{1}
&\domred: \text{Firmas} \times \SOE \rightarrow \text{Dominios PDDL}\,, \\
&\insred: \text{Firmas} \times \SOE \times \struc \rightarrow \text{Instancias PDDL}
\end{alignat*}
tal que $\domain=\domred(\sigma,\Phi)$ es un dominio PDDL
y $\instance=\insred(\sigma,\Phi,\A)$ es una instancia PDDL.

Para obtener algo de interés teórico y práctico, la reducción debe correr en
tiempo polinomial y su salida debe ser resoluble en NP para que la complejidad
de resolver su salida no sea mayor que la complejidad de resolver su entrada.
Sin embargo, el problema de decisión de existencia de un plan para PDDL no está
en NP porque (1)~el número de condiciones y acciones instanciadas puede ser
exponencial en el tamaño de la entrada, y (2~) un plan de longitud mínima puede
ser de tamaño exponencial en el número de condiciones y acciones instanciadas.
Por lo tanto, no todas las reducciones son adecuadas para este propósito y su
diseño debe realizarse cuidadosamente. En esta sección se
presenta una reducción aceptable y se estudian sus propiedades formales.

%Un problema STRIPS instanciado es una tupla $P=\tup{C,A,I,F}$
%donde $C$ es un conjunto de condiciones (proposiciones),
%$I\subseteq C$ denota el estado inicial,
%$F\subseteq C$ denota el estado final
%y $A$ es una colección de operadores.
%
%Each action $a\in O$ is defined by three subsets of fluents $pre(a)$,
%$add(a)$ and $del(a)$ that stand for the precondition, and the
%positive and negative effects of the action.
%As usual, action $a$ is applicable at state $s\subseteq F$ iff
%$pre(a)\subseteq s$, and the result of applying $a$ at such state
%is $res(s,a)=(s\setminus del(a))\cup add(a)$.
%A plan for state $s$ is a sequence of actions applicable from the
%initial state $I$ that achieves the goal condition.

%, and hence the \emph{grounding}
%of $\tup{\domain,\instance}$ results in an STRIPS problem $P$.
%The size of the grounding is polynomial for \emph{fixed} domain,
%but exponential for unrestricted domain and instance.

\subsection{Reducción del dominio}

$\domred(\sigma,\Phi)$ produce un dominio para la firma
$\sigma$ y la sentencia $\Phi\in\LSO(\sigma)$ de la forma
$(\exists R_1^{a_1})\cdots(\exists R_n^{a_n})\psi$.
Las acciones en el dominio se dividen en tres grupos: acciones para 
colocar el valor de verdad de las variables de segundo orden (relaciones
cuantificadas), acciones para probar la sentencia $\psi$ y dos otras acciones.

\subsubsection{Acciones para las variables}
Para cada variable de segundo orden $R_i$ de aridad $a_i$,
existe una acción \fttt{colocar\_verdadera\_Ri} con $a_i$ parámetros que
coloca la condición \fttt{Ri} y quita $\fttt{libre-Ri}$, el cual es
inicialmente verdadero; estas condiciones se usan para denotar el valor de
verdad de $R_i$. Por ejemplo, la acción para la relación $T^1$ en SAT es

{\footnotesize
\begin{Verbatim}
(:action set_T_true
  :parameters (?x)
  :precondition (and (guess) (not-T ?x))
  :effect (and (T ?x) (not (not-T ?x))))
\end{Verbatim}
}

\subsubsection{Acciones para las subfórmulas}
Los operadores del segundo tipo están diseñados para construir una prueba de
$\psi$ (si existe) por inducción sobre la estructura de $\psi$.
Para cada subfórmula $\theta$ de $\psi$, hay una condición
$\FT[\theta]$ que denota su validez, y hay una acción que la añade.
Los parámetros de la condición $\FT[\theta]$ son las variables libres de
$\theta$; la función $\FT[\cdot]$ se llama la \textit{traducción de la
condición}.

El primer paso en la traducción es quitar todas las implicaciones
y mover todas las negaciones hacia los literales mediante aplicaciones
repetidas de las leyes de De~Morgan. Luego, se generan las acciones recorriendo
recursivamente todas las subfórmulas $\theta$ de $\psi$ utilizando
\textit{depth-first search} de la siguiente forma:
\footnote{$\theta(\bar x)$ significa que
$\bar x$ son las variables libres de $\theta$.}

\begin{enumerate}[--]
\item si $\theta(\bar x)=\bigwedge_{i=1}^n \theta_i(\bar x_i)$
  con $\bar x=\cup_{i=1}^n \bar x_i$, entonces generar la acción
  $\prove[\theta]$ con parámetros $\bar x$, precondición
  $\bigwedge_{i=1}^n \FT[\theta_i](\bar x_i)$ y efecto \textit{add}
  $\FT[\theta](\bar x)$,
%
\item si $\theta(\bar x)=\bigvee_{i=1}^n \theta_i(\bar x_i)$ con
  $\bar x=\cup_{i=1}^n \bar x_i$, entonces generar $n$ acciones de la forma
  $\prove[\theta]_i(\bar x_i)$ con precondición
  $\FT[\theta_i](\bar x_i)$ y efecto \textit{add} $\FT[\theta](\bar x)$,
%
\item si $\theta(\bar x)=(\exists y)\theta'(\bar x,y)$, entonces generar
  $\prove[\theta](\bar x,y)$ con precondición $\FT[\theta'](\bar x,y)$
  y efecto \textit{add} $\FT[\theta](\bar x)$,
%
\item si $\theta(\bar x)=(\forall y)\theta'(\bar x,y)$ entonces generar
  dos acciones. La idea es probar $\theta(\bar x)$, variando $y$
  sobre todos los objetos.

  La primera acción $\prove[\theta]_0(\bar x)$ prueba $\theta'(\bar x,0)$.
  La acción tiene parámetros $\bar x$, precondición $\FT[\theta'](\bar x,0)$
  y efecto \textit{add} $\FT[(\forall y\leq z)\theta'(\bar x,y)](\bar x,0)$.
  (Observar que la traducción de la condición es aplicada a diferentes
fórmulas,
  en las cuales la cuantificación está acotada por $z$.)

  La segunda acción $\prove[\theta]_1(\bar x,z',z'')$ prueba inductivamente
  $(\forall y\leq z)\theta'(x,y)$ una vez que $(\forall y<z)\theta'(x,y)$ es
verdad.
  La acción tiene parámetros $\bar x,z',z''$, precondición
  $\FT[(\forall y\leq z)\theta'(\bar x, y)](\bar x,z')\land\FT[\theta'](\bar x,z'')\land \SUC(z',z'')$
  y efecto \textit{add} $\FT[(\forall y\leq z)\theta'(\bar x,y)](\bar x,z')$.
\end{enumerate}
Todas estas acciones tienen como precondición adicional la condición
\fttt{prueba}. Además, nótese que no hay acciones para las fórmulas literales.
Éstas son manejadas por la traducción de condiciones, cuya función es asignar
fórmulas a condiciones de esta manera:
\begin{enumerate}[--]
\item $\FT[Q(\bar x)](\bar x)=Q(\bar x)$,
\item $\FT[\neg Q(\bar x)](\bar x)=\fttt{no-}Q(\bar x)$,
\item $\FT[(\forall y)\theta'(\bar x,y)](\bar x) =
  \FT[(\forall y\leq z)\theta'(\bar x,y)](\bar x,\max)$,
\item en cualquier otro caso, $\FT[\theta](\bar x)=\fttt{es\_cierto\_<id>}(\bar x)$
  donde \fttt{<id>} es un identificador único para $\theta$.
\end{enumerate}
La función de traducción también se encarga de instanciar variables a términos
cuando es aplicada a literales que involucran a constantes.

\subsubsection{Otras acciones}
La traducción requiere de dos otras acciones. La primera,
llamada \fttt{empezar\_prueba} sirve para cambiar la fase de
`conjetura' a `prueba', tiene precondición \fttt{conjetura}, agrega
\fttt{prueba} y quita \fttt{conjetura}. La otra acción se llama
\fttt{probar-meta}, tiene precondición $\FT[\psi]$ y agrega 
\fttt{es\_cierto\_meta}.

%Figure~\ref{figure:sat} shows the domain for $\Phi_\SAT$.

%\begin{SaveVerbatim}{SAT}
%(define (domain SAT)
%  (:constants zero max)
%  (:predicates
%    (holds_and_2 ?x ?y) (holds_and_6 ?x0 ?x1)
%    (holds_exists_8 ?x0) (holds_forall_9 ?x0)
%    (holds_or_7 ?x0 ?x1) (holds_goal)
%    (N ?x ?y) (P ?x ?y) (T ?x) (not-T ?x)
%    (suc ?x ?y)
%  )
%  (:action set_T_true
%    :parameters (?x)
%    :precondition (and (guess) (not-T ?x))
%    :effect (and (T ?x) (not (not-T ?x))))
%
%  (:action prove_forall_9_1
%    :precondition (and (proof)
%                       (holds_exists_8 zero))
%    :effect (holds_forall_9 zero))
%
%  (:action prove_forall_9_2
%    :parameters (?y1 ?y2)
%    :precondition (and (proof)
%                       (suc ?y1 ?y2)
%                       (holds_forall_9 ?y1)
%                       (holds_exists_8 ?y2))
%    :effect (holds_forall_9 ?y2))
%
%  (:action prove_exists_8
%    :parameters (?y ?x)
%    :precondition (and (proof)
%                       (holds_or_7 ?y ?x))
%    :effect (holds_exists_8 ?y))
%
%  (:action prove_or_7_0
%    :parameters (?y ?x)
%    :precondition (and (proof)
%                       (holds_and_2 ?y ?x))
%    :effect (holds_or_7 ?y ?x))
%
%  (:action prove_or_7_1
%    :parameters (?y ?x)
%    :precondition (and (proof)
%                       (holds_and_6 ?y ?x))
%    :effect (holds_or_7 ?y ?x))
%
%  (:action prove_and_2
%    :parameters (?y ?x)
%    :precondition (and (proof)
%                       (P ?x ?y) (T ?x))
%    :effect (holds_and_2 ?y ?x))
%
%  (:action prove_and_6
%    :parameters (?y ?x)
%    :precondition (and (proof)
%                       (N ?x ?y) (not-T ?x))
%    :effect (holds_and_6 ?y ?x))
%
%  (:action prove-goal
%    :precondition (holds_forall_9 max)
%    :effect (holds_goal))
%
%  (:action begin-proof
%    :precondition (guess)
%    :effect (and (proof) (not (guess)))) )
%\end{SaveVerbatim}

%\begin{figure}
%\begin{center}
%\footnotesize
%\UseVerbatim{SAT}
%\end{center}
%\vskip -.85em
%\caption{Domain translation for $\Phi_\text{sat}=(\exists T^1)(\forall y)(\exists x)$
%$[(P(x,y)\land T(x))\lor(N(x,y)\land\neg T(x))]$.}
%\label{figure:sat}
%\end{figure}

\subsubsection{Abreviaciones}
Al haber abreviaciones, las acciones para las variables de segundo orden son
extendidos para hacer la traducción más eficiente. Para $(\exists F\in\Fun)$,
la precondición y el \textit{delete} de \fttt{colocar\_verdadera\_F} son
extendidos con la condición \fttt{(libre\_F\_dom ?x)} de modo de que pueda
haber a lo sumo una condición $F(x,y)$ verdadera para cada $x$. Así, no hay
necesidad de incluir la subfórmula $\psi_\text{fun}$. 

De forma similar, para $(\exists F\in\Inj)$ la precondición y el
\textit{delete}
son extendidos adicionalmente con la condición \fttt{(libre\_F\_ran ?y)}.

\subsection{Problema}

\label{traduccionproblema}
El problema PDDL es generado por la llamada $\insred(\sigma,\Phi,\A)$, donde
$\sigma$ es una firma, $\Phi\in\LSO(\sigma)$ y $\A\in\struc[\sigma]$.
Los objetos en el problema corresponden a los elementos en el universo
$|\A|=\{0,\ldots,n-1\}$: $0$ se asigna al objeto
\fttt{zero}, $n-1$ al objeto \fttt{max}, y los otros elementos
$0<i<n-1$ a los objetos \fttt{obj\_i}.
El objetivo es alcanzar la condición \fttt{es\_cierto\_meta}, y la situación
inicial consiste en condiciones describiendo el valor de verdad de todas las
relaciones en $\A$ y las relaciones predefinidas como $<$, SUC y otras. Sin
embargo, no es necesario incluir condiciones para relaciones que no se
mencionan en $\Phi$. Además, para cada variable de segundo orden $R$, la
situación inicial tiene condiciones que denotan que los valores de verdad de
$R$ no han sido asignados (\fttt{libre\_f}), y en casos en los que un símbolo
represente una función o función inyectiva, la situación inicial incluye
condiciones del tipo \fttt{libre\_R\_dom} y \fttt{libre\_R\_ran}.

%\subsubsection{Constantes}
%Los símbolos de constantes que pertenecen a la firma (por ejemplo,
%$K\in\sigma$) son `preprocesados' y convertidos a relaciones unarias
%que son ciertas para el elemento en $\A$ que interpreta $K$, i.e. $K^\A$.
%El preprocesamiento consiste en reescribir la 

%$K^\A$. The preprocessing consists of rewriting the logical %!fix
%formulae that use the constant into appropriate formulae that
%use the new relation, and in extending the initial situation of
%the problem with the value of the relation.

\section{Propiedades formales}
Las propiedades más importantes a considerar en la herramienta son solidez,
completitud y garantía de complejidad. Que la función de traducción sea sólida
y completa significa que ella realmente implementa una reducción entre problemas de
decisión, mientras que la garantía de complejidad se refiere al tiempo
requerido para computar la reducción y la complejidad de decidir la existencia
de un plan en el problema generado. En esta sección se demuestra que la 
herramienta es una reducción en tiempo
polinomial del problema NP expresado por $\Phi$ a un fragmento de planificación
que es decidible en NP.

%! ???
Considere la función de \emph{instanciación} $\ground$ que transforma el par
$\tup{\domain,\instance}$ de un dominio y problema PDDL en un problema STRIPS
$P = \ground(\domain, \instance)$. Para un $\domain$ fijo, la función
$\instance\leadsto\ground(\domain, \instance)$ corre en un tiempo polinomial
$\O(\|\instance\|^k)$ para algún $k$ que sólo depende de $\domain$; de hecho,
$k$ es la máxima aridad de un predicado o acción en el dominio.

De manera similar, la función de traducción $\insred$ corre en tiempo
polinomial en el tamaño de la estructura $\A$, pero exponencial sobre la aridad
más grande de un cuantificador de segundo orden existencial en $\Phi$. De este
modo, la función $f_{\sigma,\Phi}:\struc[\sigma]\rightarrow\text{STRIPS}$
definida por 
\[ f_{\sigma,\Phi}(\A) = \ground(\domred(\sigma,\Phi),\insred(\sigma,\Phi,\A)) \]
es una función en tiempo polinomial que asigna a las $\sigma$-estructuras un
problema instanciado STRIPS.

Se debe demostrar que la función $f_{\sigma,\Phi}(\A)$ es una reducción y que
los problemas de planificación que produce tienen a lo sumo la misma
complejidad que \SOE.

\subsection{Demostración de reducción}
Se demostrará que $\A \models \Phi$ si y sólo si $f_{\sigma,\Phi}(\A)$ tiene
solución, es decir, si y sólo si existe un plan que alcanza $\FT[\Phi]$ desde
el estado inicial (que depende de $\A$).

\begin{proposition}
\label{t1}
Sean $t_1,\ldots,t_k$ términos instanciados, es decir, que no contienen
variables. Sea $\A$ una estructura, $u\in|\A|$ y `$a$' una constante tal que
$a^\A=u$.
Entonces, $(\A, i[x:=u]) \models \varphi(t_1,\ldots,t_k,x)$ si y sólo si
$(\A, i) \models \varphi(t_1,\ldots,t_k,a)$.
\end{proposition}
\begin{proof}
Directa de la definición de verdad.
\end{proof}
\begin{corollary}
\label{corolario}
Sean $t_1,\ldots,t_k$ términos instanciados. Sea $\A$ una
estructura en donde todos los objetos tienen nombre (i.e., para todo $u\in|\A|$
existe una constante $a$ tal que $a^\A=u$). Entonces,
$(\A, i) \models (\exists x) \varphi(t_1,\ldots,t_k,x)$ si y sólo si
existe una constante `$a$' tal que $(\A, i) \models \varphi(t_1,\ldots,t_k,a)$.
\end{corollary}

\begin{theorem}
\label{teoremaprimerorden}
Sea $\varphi \in \LPO(\sigma)$ y $t_1,\ldots,t_k$ $\sigma$-términos instanciados. Sea $\A \in
\struc[\sigma]$ tal que todo objeto tiene nombre y sea $i : \text{VAR} \longrightarrow |\A|$.
Entonces, $(\A, i) \models \varphi(t_1,\ldots,t_k)$ si y sólo si
existe un plan $\pi[t_1,\ldots,t_k]$ que alcanza
$\FT[\varphi](t_1,\ldots,t_k)$ desde el \textbf{estado inicial} $S[\A]$,
donde $S[\A]$ es el estado de STRIPS en el cual los operadores que construyen la
prueba empiezan a ser aplicables, definido por las siguientes condiciones:
\begin{alignat*}{1}
& S[\A] = \{\texttt{prueba}\} \cup \{ \texttt{R}(a_1,\ldots,a_k) :
\tup{a_1^\A,\ldots,a_k^\A} \in R^\A\}\\
& {\color{white} S[\A] = } \cup \{\texttt{no-R}(a_1,\ldots,a_k) : \tup{a_1^\A,\ldots,a_k^\A} \not\in R^\A\}\\
& {\color{white} S[\A] = } \cup \{\texttt{SUC}(a_i, a_{i+1}) : 0 \leq i < n\}
\end{alignat*}
\end{theorem}
\begin{proof}
Hacemos la demostración por inducción sobre la estructura de $\varphi$. El caso
base corresponde a las fórmulas atómicas de la forma $\varphi =
R(t_1,\ldots,t_k)$:
\begin{alignat*}{1}
& \quad \quad (\A, i) \models \varphi(t_1,\ldots,t_k)\\
& \iff \tup{\text{definición \ref{semantica_def}(b)}} \\
& \quad \quad \tup{t_1,\ldots,t_k} \in R^\A\\
& \iff \tup{\text{definición de S[\A]}}\\
& \quad \quad \texttt{R}(t_1,\ldots,t_k) \in S[\A]\\
& \iff \tup{\text{definición de } \FT}\\
& \quad \quad \FT[\varphi](t_1,\ldots,t_k) \in S[\A]\\
& \iff \tup{\text{el plan vacío alcanza la condición } \FT[\varphi](t_1,\ldots,t_k)}\\
& \quad \quad \pi[t_1,\ldots,t_k] =\ \tup{} \textbf{ es el plan.}
\end{alignat*}
El caso $\varphi = \neg R(t_1,\ldots,t_k)$ es análogo al anterior, con
\texttt{not-R}$(t_1,\ldots,t_k) \in S[\A]$ en lugar de \texttt{R}$(t_1,\ldots,t_k)$.
Ahora consideramos las fórmulas más complejas:

Caso $\varphi = \varphi_1(t'_1,\ldots,t'_{k'}) \land \varphi_2(t_1'',\ldots,t''_{k''}) $:
\begin{alignat*}{1}
& \quad \quad (\A, i) \models \varphi(t_1,\ldots,t_k)\\
& \iff \tup{\text{definición \ref{semantica_def}(d)}}\\
& \quad \quad (\A, i) \models \varphi_1(t'_1,\ldots,t'_{k'}) \text{ y } 
(\A, i) \models \varphi_2(t''_1,\ldots,t''_{k''})\\
& \iff \tup{\text{hipótesis inductiva}}\\
& \quad \quad \pi_1[t'_1,\ldots,t'_{k'}] \text{ y } 
\pi_2[t_1'',\ldots,t''_{k''}] \text{ son planes que alcanzan }\\
& \quad \quad \FT[\varphi_1](t'_1,\ldots,t'_{k'}) \text{ y } \FT[\varphi_2](t'_1,\ldots,t'_{k''})
\text{ desde } S[\A]\\
& \iff \langle \text{como no hay \textit{deletes}, aplicar ambos planes en
secuencia }\\
& {\color{white}\iff\langle} \text{garantiza alcanzar
$\FT[\varphi](t_1,\ldots,t_k)$ desde $S[\A]$} \rangle\\
& \quad \quad \textbf{el plan }\tup{\pi_1(t'_1,\ldots,t'_{k'});\
\pi_2(t_1'',\ldots,t_{k_2}'');\ \texttt{probar\_y\_}\varphi(t_1,\ldots,t_k)} \\
& \quad \quad \textbf{alcanza la condición }
\FT[\varphi](t_1,\ldots,t_k) \textbf{ desde } S[\A]
\end{alignat*}

Caso $\varphi = \varphi_1(t_1',\ldots,t'_{k_1}) \lor \varphi_2(t''_1,\ldots,t''_{k_2}) $:
\begin{alignat*}{1}
& \quad \quad (\A, i) \models \varphi(t_1,\ldots,t_k)\\
& \iff \tup{\text{definición \ref{semantica_def}(e)}}\\
& \quad \quad (\A, i) \models \varphi_1(t'_1,\ldots,t'_{k'}) 
\text{ o } (\A, i) \models \varphi_2(t''_1,\ldots,t''_{k''})\\
& \text{Por casos. Caso $(\A, i) \models \varphi_1(t'_1,\ldots,t'_{k'})$:}\\
& \quad \iff \tup{\text{hipótesis inductiva}}\\
& \quad \quad \quad \pi_1[t_1,\ldots,t_{k_1}] \text{ es un plan que alcanza 
$\FT[\varphi_1](t_1,\ldots,t_{k_1})$ desde $[\A]$}\\
& \quad \iff \langle \text{no hay \textit{deletes}} \rangle\\
& \quad \quad \textbf{el plan } \tup{\pi_1(t_1,\ldots,t_{k_1});\
\texttt{probar\_o\_}\varphi(t_1,\ldots,t_k)} \\
& \quad \quad \textbf{alcanza la condición }
\FT[\varphi](t_1,\ldots,t_k) \textbf{ desde } S[\A]\\
& \text{Caso $(\A, i) \models \varphi_2(t''_1,\ldots,t''_{k''})$:}\\
& \quad \quad \text{Análogo al caso anterior.}
%Caso $\varphi = \varphi_1(t_1',\ldots,t'_{k'}) \Rightarrow \varphi_2(t''_1,\ldots,t''_{k''}) $:
%\quad Ver caso anterior, pues $\varphi_1 \Rightarrow \varphi_2 = \neg \varphi_1 \lor \varphi_2$.
\end{alignat*}

Caso $\varphi = (\exists x)\ \theta(t_1,\ldots,t_k,x):$
\begin{alignat*}{1}
& \quad \quad (\A, i) \models \varphi\\
& \iff \langle \text{por corolario \ref{corolario}, ya que todo objeto en $|\A|$
tiene nombre, }\\
& {\color{white} \iff \langle} \text{ existe una constante } `a' \rangle\\
& \quad \quad (\A, i) \models \theta(t_1,\ldots,t_k,a)\\
& \iff \tup{\text{hipótesis inductiva}}\\
& \quad \quad \pi[t_1,\ldots,t_k,a] \text{ es un plan que alcanza }
\FT[\theta](t_1,\ldots,t_k,a) \text{ desde } S[\A]\\
& \iff \tup{\text{no hay \textit{deletes}}}\\
& \quad \quad \textbf{el plan } \langle \pi[t_1,\ldots,t_k,a];\
\texttt{probar\_existencial\_}\varphi(t_1,\ldots,t_k)\rangle\\
& \quad \quad \textbf{alcanza la condición } \FT[\varphi](t_1,\ldots,t_k)
\textbf{ desde } S[\A]
\end{alignat*}

Caso $\varphi = (\forall x)\ \theta(t_1,\ldots,t_k,x):$
\begin{alignat*}{1}
& \quad \quad (\A, i) \models \varphi\\
& \iff \tup{\text{definición \ref{semantica_def}(h), y todo objeto tiene nombre}}\\
& \quad \quad (\A, i) \models \theta(t_1,\ldots,t_k,a), \text{ para toda
constante $a$}\\
& \iff \tup{\text{hipótesis inductiva}}\\
& \quad\quad \text{para toda constante $a$ existe un plan $\pi[t_1,\ldots,t_k,a]$ que}\\
& \quad\quad \text{alcanza $\FT[\theta](t_1,\ldots,t_k,a)$ desde } S[\A]\\
& \iff \langle \text{existe un orden $a_i < a_{i+1}$ para todas las constantes,}\\
& {\color{white} \iff \langle} \text{ determinado por \texttt{SUC}, y no hay
\textit{deletes}} \rangle\\
& \quad \quad \textbf{el plan } \langle \pi_0[t_1,\ldots,t_k,a_0];\
\texttt{probar\_universal\_base\_}\varphi(t_1,\ldots,t_k);\\
& \quad \quad \pi_1[t_1,\ldots,t_k,a_1];\
\texttt{probar\_universal\_inductivo\_}\varphi(t_1,\ldots,t_k,a_1);\\
& \quad \quad \ldots\ \\
& \quad \quad \pi_{n-1}[t_1,\ldots,t_k,a_{n-1}];\
\texttt{probar\_universal\_inductivo\_}\varphi(t_1,\ldots,t_k,a_{n-1}) \rangle\\
& \quad \quad \textbf{alcanza la condición } \FT[\varphi](t_1,\ldots,t_k)
\textbf{ desde } S[\A]
\end{alignat*}
\end{proof}

Se define ahora formalmente el estado inicial de un problema de planificación
generado por la traducción. La traducción completa del
problema está en la sección \ref{traduccionproblema}.

\begin{definition}
Sea $\Phi = (\exists R_1^{k_1})\cdots(\exists R_m^{k_m}) \varphi$.
\begin{alignat*}{1}
& \textsc{init}(f_{\sigma,\Phi}(\A)) = \{\{ \texttt{P}(a_1,\ldots,a_k) :
\tup{a_1^\A,\ldots,a_k^\A} \in P^\A\}\\
& {\color{white} \texttt{init}(f_{\sigma,\Phi}(\A)) = } \cup \{\texttt{no-P}(a_1,\ldots,a_k) : \tup{a_1^\A,\ldots,a_k^\A} \not\in P^\A\}\\
& {\color{white} \texttt{init}(f_{\sigma,\Phi}(\A)) = } \cup \{\texttt{SUC}(a_i, a_{i+1}) : 0 \leq i < n \} \\
& {\color{white} \texttt{init}(f_{\sigma,\Phi}(\A)) = } \cup
\{\texttt{libre-R\_i}(t)) : \text{ para todo $t \in |\A|^{k_i}$},
R_i\in\{R_1,\ldots,R_m\}\} \}
\end{alignat*}
\end{definition}

Finalmente, puede probarse que la función es una reducción:
\begin{theorem}
Sea $\Phi = (\exists R_1)\cdots(\exists R_n) \varphi$.
$\A \models \Phi$ si y sólo si existe un plan $\pi$ que alcanza $\FT[\Phi]$
desde el estado inicial $\textsc{init}(f_{\sigma,\Phi}(\A))$.
\end{theorem}
\begin{proof}
Por definición, $\A \models \Phi$ si y sólo si
$\tup{\A, R_1',\ldots,R_n'} \models \varphi$.
Por Teorema \ref{teoremaprimerorden}, esto es verdad si y sólo si 
existe un plan $\pi'$ que alcanza $\FT[\varphi]$ a partir de $S[\A,
R_1',\ldots,R_n']$.
Basta mostrar que es posible alcanzar el estado $S[\A, R_1',\ldots,R_n']$ desde
$\textsc{init}(f_{\sigma,\Phi}(\A))$. Sea $u_j = a_j^\A$ para $1 \leq j \leq k$.
\begin{alignat*}{1}
& \textbf{El plan }\\
& \langle \ \tup{\texttt{colocar\_verdadera\_Ri}(a_1,\ldots,a_{k_i}) 
\text{ para cada } \tup{u_1,\ldots,u_k} \in R_i} \text{ para cada } R_i \ ;\\
& {\color{white} \langle \ } \tup{\texttt{colocar\_falsa\_Ri}(a_1,\ldots,a_{k_i}) \text { para cada
} \tup{u_1,\ldots,u_k} \not\in R_i} \text{ para cada } R_i\ ;\\
& {\color{white} \langle \ } \texttt{comenzar\_prueba}()\ \rangle\\
& \textbf{alcanza el estado } S[\A, R_1',\ldots,R_n'] \textbf{ desde }
\textsc{init}(f_{\sigma,\Phi}(\A)).
\end{alignat*}
\end{proof}

\subsection{Demostración de complejidad}
Como se ha mencionado en la sección \ref{complejidad_planificacion}, 
se sabe que verificar la existencia de un plan para 
problemas de planificación
sin efectos negativos está en NP \citep{bylander:plan-complexity}. La prueba
depende del hecho de que un plan óptimo no necesita repetir acciones, y por lo
tanto es de tamaño lineal. Puede extenderse esta noción: si todas las acciones que
agregan efectos negativos pueden ser aplicadas a lo sumo una vez, verificar la
existencia de un plan sigue estando en NP, pues el tamaño del plan no podrá ser
mayor al número de acciones.

\begin{definition}
Un problema de planificación $P = \tup{C, A, I, F}$ es de tipo
\textit{máximo-1} si y sólo si las acciones pueden ser particionadas en $A =
A_0 \cup A_1$, donde:
\begin{itemize}
\item Ninguna de las acciones de $A_0$ tiene efectos negativos, es decir,
$(\forall a \in A_0) (del(a) = \emptyset)$.
\item Todas las acciones de $A_1$ tienen una precondición que no es añadida por
ninguna acción, y que es borrada apenas $a$ es ejecutada, es decir, 
$(\forall aa' \in A_1)\ (\exists p \in pre(a) \cap del(a))\ (p \not\in add(a'))$
\end{itemize}

El conjunto de todos los problemas \textit{máximo-1} se denota como STRIPS-1.
\end{definition}

\begin{theorem}
Para toda estructura $\A$, $f_{\sigma, \Phi}(\A)$ es un problema STRIPS-1.
\end{theorem}
\begin{proof}
Ninguna de las acciones de $f_{\sigma, \Phi}(\A)$ tienen \textit{deletes} excepto
\texttt{comenzar\_prueba} y las acciones \texttt{colocar\_verdadera} y
\texttt{colocar\_falsa}, pero todas estas borran una precondición que no es
añadida por ninguna otra acción (i.e., pertenecen a $A_1$ según la definición
de STRIPS-1).
\end{proof}

\begin{theorem}
El problema de decisión de existencia de un plan STRIPS-1 es NP-completo.
\end{theorem}
\begin{proof}
\ \\ \textbf{Inclusión:} Todas las acciones de $A_1$ pueden ser aplicadas a lo
sumo una vez. Como estas acciones pueden borrar condiciones (tienen
\textit{deletes}), en el peor caso puede requerirse aplicar \textbf{todas} las
acciones de $A_0$ antes de aplicar otra acción de $A_1$. Por lo tanto, el
tamaño de un plan es a lo sumo cuadrático en el número total de acciones.
\\ \textbf{Dificultad:} $f_{\sigma_{\textsc{SAT}}, \Phi_{\textsc{SAT}}}$ reduce SAT a STRIPS-1 en
tiempo polinomial.
\end{proof}

Por lo tanto, la herramienta traduce cualquier problema NP, codificado con una
sentencia \SOE, a STRIPS-1.

\section{Diseño del lenguaje}

\subsection{Gramática}
\begin{alignat*}{1}
\sofbf\   & \deriv\ \lpar \SOExists\ \lpar \listrel \rpar\ \sofbf \rpar \\
          & \deriv\ \lpar \SOForall\ \lpar \listrel \rpar\ \sofbf \rpar \\
          & \deriv\ \pofbf \\[1em]
\listrel\ & \deriv\ \rel\ \integer\ \listrel\ |\ \rel\ \integer \\
          & \deriv\ \rel\ \tipof\ \listrel\ |\ \rel\ \tipof\ \\[1em]
\tipof\   & \deriv\ \Fun\ |\ \PFun\ |\ \Inj\ |\ \PInj\ \\[1em]
\pofbf\   & \deriv\ \lpar \rel\ \listvar \rpar \\
          & \deriv\ \lpar \NOT\ \pofbf \rpar \\
          & \deriv\ \lpar \AND\ \listpofbf \rpar \\
          & \deriv\ \lpar \OR\ \listpofbf \rpar \\
          & \deriv\ \lpar \IMPLIES\ \pofbf\ \pofbf \rpar \\
          & \deriv\ \lpar \Exists\ \lpar \listvar \rpar\ \pofbf \rpar \\
          & \deriv\ \lpar \Forall\ \lpar \listvar \rpar\ \pofbf \rpar \\[1em]
\listvar\ & \deriv\ \var\ \listvar\ |\ \var \\[1em]
\end{alignat*}


%\begin{theorem}
%La función $f_{\sigma, \Phi}$ es una reducción en tiempo polinomial del
%problema de decisión inducido por $\Phi$ a STRIPS-1.
%\end{theorem}
%
%\begin{proof}
%Se quiere demostrar que
%$\A\models(\exists R_1^{a_1})\cdots(\exists R_k^{a_k})\psi$ si y sólo si
%existe un plan que consigue la condición $\FT[\psi]$. La prueba es inductiva
%sobre la estructura de la sentencia de primer orden $\psi$. Se demostrará que para toda subformula 
%$\theta$ de $\psi$, y toda interpretacion $(\\A,i)$ de $\A$:
%
%\[ (\\A,i) \models \theta(\overline{X}) \iff \text{existe un plan para } \FT[\theta](\overline{X}) \]
%
%Por implicación mutua. Primero se prueba
%\[ (\\A,i) \models \theta(\overline{X}) \Rightarrow \text{existe un plan para } \FT[\theta](\overline{X}) \]
%\end{proof}

% Chapter 4

\chapter{Traducción de problemas PH}
\label{Chapter3}
\lhead{Capítulo 3. \emph{Traducción de problemas PH}}

% Chapter 4

\chapter{Traducción de problemas PH}
\label{Chapter4}
\lhead{Capítulo 4. \emph{Traducción de problemas PH}}

% Chapter 5
\chapter{Conclusiones y Recomendaciones}
\label{Capitulo5}
\lhead{\emph{Conclusiones y Recomendaciones}}

Este trabajo ha presentado dos reducciones de la lógica de segundo orden a
problemas de planificación automática que se adhieren a la Teoría de
Complejidad Descriptiva. Las herramientas reciben como entrada una firma
$\sigma$, una sentencia en lógica de segundo orden (segundo orden existencial,
en el caso de la traducción de NP) y una estructura $\A\in\struc[\sigma]$, y
producen como salida un par de archivos en PDDL $\tup{\domain,\instance}$, que
representan un dominio y una instancia de planificación en STRIPS. La
conversión corre en tiempo polinomial en el tamaño de la estructura $\|\A\|$, y
por lo tanto puede usarse como un método eficiente para generar reducciones de
problemas NP a STRIPS. Esta salida puede utilizarse con un planificador 
para obtener una solución al problema original.

Los resultados de los experimentos muestran que la herramienta puede utilizarse
para crear solucionadores a problemas específicos NP o PH que tienen un buen desempeño
sobre problemas pequeños. Así, por ejemplo, un profesional del área de
computación que tenga la necesidad de resolver problemas pequeños de grafos, o
requiera realizar pruebas de concepto para estudiar problemas académicos o de la vida
real, puede seguir el siguiente proceso para conseguir una solución rápida que
cumpla con sus objetivos:
\begin{enumerate}
\item Comparar el problema de interés con problemas de decisión conocidos en el
área de teoría de grafos, optimización, u otras.
\item Modelar el problema de interés utilizando lógica de segundo orden, y
escribirlo en la sintaxis propuesta por este trabajo.
\item Utilizar la herramienta para producir el dominio y distintas instancias
en PDDL, usar un planificador para conseguir soluciones a estas instancias y
analizar las soluciones arrojadas para constatar cómo se interpretan en el contexto del
problema de interés.
\end{enumerate}

Debe notarse que no todos los problemas son fácilmente modelables utilizando un
lenguaje descriptivo como el que hemos desarrollado en este trabajo. Así como
no existe un lenguaje de programación \textbf{ideal} que permita la expresión
de cualquier problema de una forma sencilla, sino que existen distintos
\textbf{paradigmas} y el programador debe tener la experticia para reconocer
cuál es el más adecuado para cada problema determinado, no hay una única forma
``óptima'' de tratar con todos los problemas de decisión.
Por ejemplo, sería particularmente engorroso intentar
modelar el problema de \textit{subset sum} (\textbf{suma de subconjuntos}),
utilizando el lenguaje lógico descrito, a pesar de que es un problema
NP-completo, pues se debe simular un sistema formal
aritmético que permita la suma de números en tal lógica. La teoría establece
que es posible, pero no se recomienda el uso de la herramienta para ello ya que
el ejercicio puede tomar mucho tiempo.

Además, la herramienta es de interés para los investigadores del área de
planificación automática, pues los problemas de planificación que genera
presentan un reto para los planificadores actuales y pueden utilizarse como una
medida de comparación entre diferentes enfoques a cómo construir planificadores
generales y eficientes, particularmente en el caso de \textit{SAT-planners} y
en la búsqueda de planes paralelos.

La reducción a NP presentada en ese trabajo fue presentada en el mes de junio
de 2011 en la Conferencia Internacional de
Planificación Automática (ICAPS '11) en la ciudad de Freiburg, Alemania.
La extensión a la herramienta que le permite procesar problemas en PH ha sido
propuesta como artículo de investigación corto en ICAPS '13, conferencia que
tendrá lugar en Roma, Italia. Se recibirán noticias sobre su aceptación en
enero de 2013.

Para seguir adelante con esta línea de investigación se proponen las siguientes
recomendaciones:

\begin{enumerate}[--]
\item Diseñar una segunda extensión a la herramienta, modificando el lenguaje
para incorporar el formalismo de \textbf{clausura transitiva}, necesario para
resolver problemas generales de la clase de complejidad PSPACE. Con tal
extensión, el poder expresivo del lenguaje presentado aquí se equipararía a
PDDL.
\item Optimizar las reducciones para distintos tipos de planificadores, ya que
el trabajo se concentró en planfificadores basados en SAT. Como punto de
partida, puede investigarse a fondo por qué el planificador LAMA '11 tuvo
mejores resultados que M en el dominio \coCOL.
\end{enumerate}


%% ----------------------------------------------------------------
%the Bibliography 
\label{Bibliography}
\bibliography{bibliografia}  % The references (bibliography) information are stored in the file named "Bibliography.bib"
\lhead{\emph{Bibliografía}}  % Change the left side page header to "Bibliography"
\bibliographystyle{aaai}  % Use the "unsrtnat" BibTeX style


%% ----------------------------------------------------------------
% Now begin the Appendices, including them as separate files

\addtocontents{toc}{\vspace{2em}} % Add a gap in the Contents, for aesthetics

\appendix % Cue to tell LaTeX that the following 'chapters' are Appendices

% Appendix A

\chapter{Problemas NP y PH modelados en lógica}
\label{apendiceA}
\lhead{Apéndice A. \emph{Problemas NP y PH modelados en lógica}}

\section{Problemas NP}

\subsection{SAT, Satisfacibilidad proposicional}
Existe una asignación de verdad $T$, tal que para toda cláusula $y$, existe una variable $x$, tal que:
$x$ está positiva en $y$ y $T(x)$, o $x$ está negativa en $y$ y $not-T(x)$,
\begin{verbatim}
(so-exists (?T 1)
  (forall (?y)
    (exists (?x)
      (or
        (and (?P ?x ?y)
             (?T ?x)) 
        (and (?N ?x ?y)
             (not (?T ?x)))))))
\end{verbatim}

\subsection{CLIQUE, Clique de tamaño $k$}

En esta fórmula se utiliza la relación unaria $K$ para modelar la constante $k$
del problema. Si se quiere que $k$ sea igual a $3$, por ejemplo, se agrega a la
instancia \texttt{?K zero}, \texttt{?K obj1} y \texttt{?K obj2}.

\begin{verbatim}
(so-exists (?F Inj)
    (forall (?x) (forall (?y)
      (implies (and (< ?x ?y)
                    (exists (?z) (and (?F ?x ?z) (?K ?z)))
                    (exists (?z) (and (?F ?y ?z) (?K ?z))))
               (?E ?x ?y)))))
\end{verbatim}
\begin{figure}[h!]
\centering
\includegraphics[width=\textwidth]{figuras/clique.pdf}
\caption[Grafo con una \textit{clique} de tamaño $k = 6$]{Grafo con una clique de tamaño
$k = 6$}
\label{clique}  
\end{figure}

\subsection{\CHD, Camino Hamiltoniano Dirigido}
Esta fórmula expresa que existe una manera ordenada $F(a,b)$ de visitar los nodos de grafo (la posición $a$ la tiene el nodo $b$), en
la que para toda posición $x$, si no es la última, entonces existe un nodo $y_1$ que le corresponde esta posición, y además
a la siguiente posición $x_2$ le corresponde otro nodo $y_2$, y hay un arco entre
$y_1$ y $y_2$.
\begin{verbatim}
(so-exists (?F Inj)
  (forall (?x)
    (implies (< ?x MAX)
             (exists (?y1)
               (and (?F ?x ?y1)
                 (exists (?y2)
                   (and
                     (exists (?x2) 
                       (and (SUC ?x ?x2) (?f ?x2 ?y2)))
                   (?E ?y1 ?y2))))))))
\end{verbatim}
% \begin{verbatim}
% ; (exists F in Inj)(forall x)[x < MAX -> E(f(x),f(x+1))]
% \end{verbatim}

\begin{figure}[h!]
\centering
\includegraphics[width=\textwidth]{figuras/chd.pdf}
\caption[Grafo con \textit{camino hamiltoniano}]{Grafo con \textit{camino hamiltoniano}}
\label{hamil}  
\end{figure}

\subsection{\TDM, \textit{3-Dimensional Matching}}
Esta fórmula expresa que existen dos funciones biyectivas $F(x,y)$ y $G(y,z)$, que por 
transitividad entre ellas simulan un subconjunto de tripletas $x$-$y$-$z$ de  $T(x,y,z)$,
que no comparten ninguno de sus elementos (propiedad lograda la biyectividad de
dichas funciones).
\begin{verbatim}
(so-exists (?F Inj)
  (so-exists (?G Inj)
    (and
      (forall (?x) (exists (?y) (?F ?x ?y))) ; F is total
      (forall (?x) (exists (?y) (?G ?x ?y))) ; G is total
      (forall (?x) (forall (?y) (forall (?z)
        (implies (and (?F ?x ?y) (?G ?x ?z))
                 (?T ?x ?y ?z))))))))
\end{verbatim}

\begin{figure}[h!]
\centering
\includegraphics[totalheight=0.4\textheight]{figuras/3dimmat.png}
\caption[Ejemplo de \textit{3-Dimensional Matching}, \citep{wiki:3dm}]{Ejemplo de \textit{3-Dimensional Matching}, \citep{wiki:3dm}}
\label{mat3}  
\end{figure}

\subsection{\TCOL, 3-colorabilidad}
Esta fórmula expresa que existe una asignación de colores $R$, $G$ y $B$ tal que para todos los nodos
de un grafo, no hay dos vertices adyacentes del mismo color, los vertices tienen
almenos y a lo sumo un color.
\begin{verbatim}
(so-exists (?R 1) (so-exists (?G 1) (so-exists (?B 1)
  (forall (?x) 
    (and
      ; no hay dos vértices adyacentes del mismo color
      (forall (?y)
        (implies (?E ?x ?y) (not (or (and (?R ?x) (?R ?y))
                                        (and (?G ?x) (?G ?y))
                                        (and (?B ?x) (?B ?y))))))

        ; los vértices tienen al menos un color
        (or (?R ?x) (?G ?x) (?B ?x))

        ; los vértices tienen a lo sumo un color
        (implies (?R ?x) (and (not (?G ?x)) (not (?B ?x))))
        (implies (?G ?x) (and (not (?R ?x)) (not (?B ?x))))
        (implies (?B ?x) (and (not (?R ?x)) (not (?G ?x)))))))))
\end{verbatim}

\begin{figure}[h!]
\centering
\includegraphics[width=\textwidth]{figuras/coloring.pdf}
\caption[Grafo con \textit{3 Colorabilidad}]{Grafo 3-coloreable}
\label{col3}  
\end{figure}

\subsection{\KCOL, k-colorabilidad}
Esta fórmula expresa que existe una función de asignación de colores $F$, tal que a todo nodo se le es
asignado un color $y$, donde $K(y)$ quiere decir que el color $y$ es menor al máximo color
$k$; ademá, para todos los otros nodos, si este esta conectado a ellos, no
tienen el mismo color.

\begin{verbatim}
(so-exists (?F Func)
  (forall (?x) 
    (and (exists (?y) (and (?F ?x ?y) (?K ?y)))
      (forall (?y) 
        (implies (?E ?x ?y)
                 (not (exists (?z)
                   (and (?F ?x ?z) (?F ?y ?z)))))))))
\end{verbatim}

\section{Problemas PH}

\subsection{UNSAT, No-Satisfacibilidad proposicional}
Esta fórmula expresa que para toda relación $T$ sobre variables
proposicionales, existe una cláusula $y$ tal que para cada variable $x$, o $x$
aparece positiva en $y$ y $x$ es falsa, o $x$ aparece negativa en $y$ y $x$ es
verdadera, o $x$ no aparece en $y$.
\begin{verbatim}
(so-forall (?T 1 @ist)
    (exists (?y @cls)
        (forall (?x @var)
            (or (and (?N ?x ?y)
                     (?T ?x)
                )
                (and (?P ?x ?y)
                    (not (?T ?x))
                )
                (?NotIn ?x ?y)
             ))))
\end{verbatim}

\subsection{\qEA-QBF, Fórmula \textit{booleana} cuantificada $\Sigma_p^2$}
Esta fórmula expresa que existe una relación \texttt{E0} sobre variables
proposicionales, tal que para toda relación \texttt{A0} sobre variables
proposicionales, para toda cláusula $c$ existe una variable $x$, tal que: o $x$ aparece
positiva en $c$ y es verdadera existencial o universalmente, o $x$ aparece negativa en $c$ y
es negativa existencial o universalmente.
\begin{verbatim}
(so-exists (?E0 1 @ise0)
    (so-forall (?A0 1 @isa0)
        (forall (?c @cls)
            (exists (?x @var)
                (or
                    (and (?P ?x ?c) (?E0 ?x))
                    (and (?P ?x ?c) (?A0 ?x))
                    (and (?N ?x ?c) (not (?E0 ?x)))
                    (and (?N ?x ?c) (not (?A0 ?x)))
                )))))
\end{verbatim}

\subsection{\coCOL, No-3-Colorabilidad}
Esta fórmula expresa que para toda posible coloración de nodos, donde los
colores son expresados con las relaciones $R$ y $G$, o dos nodos unidos por un arco tienen
el mismo color, o un nodo tiene dos colores al mismo tiempo.
\begin{verbatim}
(so-forall (?R 1 @node)
    (so-forall (?G 1 @node)
        (exists (?x @node)
            (or 
                (exists (?y @node)
                    (or (and (?E ?x ?y) (?R ?x) (?R ?y))
                        (and (?E ?x ?y) (?G ?x) (?G ?y))
                        (and (?E ?x ?y) (not (?R ?x)) 
                             (not (?R ?y)) (not (?G ?x)) 
                             (not (?G ?y)))
                    )
                )
                (and (?R ?x) (?G ?x))))))
\end{verbatim}

\begin{figure}[h!]
\centering
\includegraphics[totalheight=0.4\textheight, angle=90]{figuras/clique4.pdf}
\caption[Ejemplo de grafo \textit{No-3-Coloreable}]{Ejemplo de grafo No-3-Coloreable}
\label{noclique}  
\end{figure}

% Appendix A

\chapter{Figuras adicionales}
\label{apendiceB}
\lhead{Apéndice B. \emph{Figuras adicionales}}

\begin{landscape}
\begin{figure}[h]
\centering
\includegraphics[height=\textheight]{figuras/arbolsintaxis.pdf}
\caption[Arbol sintáctico de $\Phi_{SAT}$]{Árbol sintáctico que construye el \textit{parser} al analizar la
sentencia $\Phi_{SAT}$. Los nodos terminales se marcan con un óvalo.}
\label{arbolsintactico}
\end{figure}
\end{landscape}


\addtocontents{toc}{\vspace{2em}}  % Add a gap in the Contents, for aesthetics
\backmatter

\end{document}
