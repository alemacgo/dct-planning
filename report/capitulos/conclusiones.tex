% Chapter 5
\chapter{Conclusiones y Recomendaciones}
\label{Capitulo5}
\lhead{\emph{Conclusiones y Recomendaciones}}

We presented a ``black box'' that given as input a
signature $\sigma$, a second-order existential sentence
$\Phi$ and a structure $\A\in\struc[\sigma]$, outputs
a STRIPS problem $P$ that has a plan iff $\A\models\Phi$.
The black box is fully automated and runs in polynomial
time in the size $\|\A\|$, and thus can be thought of
as an efficient method to generate polytime reductions
from NP into STRIPS.

STRIPS however is more general than NP, it is PSPACE-complete.
Thus, for any decision problem in PSPACE there is a polytime
reduction from it into STRIPS.
Currently, we do not know if such reductions can be %!fix
automatically and efficiently constructed as in the case
for NP.  

In the future we plan to work on this problem.
In order to illustrate some of the subtleties that
lie ahead, consider the following ``complementation problem''.
Since PSPACE is closed under complementation, there is a
polytime reduction $f:\text{STRIPS}\rightarrow\text{STRIPS}$
that given as input a problem $P$ outputs problem $P'=f(P)$
such that $P'$ has solution iff $P$ does not.
What does $f$ do to $P$? At this moment, we do not know
and this understanding seems necessary to be able to
construct automated reductions from PSPACE into STRIPS.

On the side of DCT, PSPACE corresponds to second-order logic
extended with transitive closures. Hence, we will not only
have to deal with the complementation problem but also with
the computation of transitive closures for predicates definable
in second-order logic.

Finally, DCT presents a crisp syntactic separation between 
complexity classes such as PSPACE, the levels of the polynomial-time
hierarchy (PH) and PH itself; e.g., $\Sigma^p_2=\SOEA$ and $\PH=\SO$.
%$\Pi^p_3=\SOAEA$ and $\PH=\SO$.
But on the side of planning, we do not know of any syntactic
or structural characterization of STRIPS that puts the
plan-existence problem in, say, $\Sigma^p_2$.
Thus, in the case that we find a method for generating reductions
from, say, \SOEA\ into STRIPS, it would most likely generate
unrestricted STRIPS problems whose complexity is in PSPACE.
Hence, further knowledge about the complexity of STRIPS 
planning awaits to be discovered \ldots

We have extended PMB's tool with a type system and over
arbitrary SO formulas that capture PH:
given such a $\Phi$ and a structure $\A$, the tool
generates a planning problem $P$ that has solution iff $\A\models\Phi$.
Since problems in PH has no short certificates in general, the solutions
for the STRIPS problems have exponential length in general.
Thus, we do not expect that planners based on forward search in state space
will be able to succeed unless heuristics that account for the multiple
application of actions are used.
However, the results for LAMA'11 on \coCOL contradict this expectation and
thus deserve an explanation. Indeed, we conjecture that the reason for
the success of LAMA'11 on these instances is the \emph{implicit serialization}
of subgoals that is achieved as a side effect of the multiple (open) queues
used by LAMA'11 \cite{richter:lama,lipovetzky:ecai12-width}.

The extension over \PSPACE is now foreseeable as \PSPACE equals SO plus
transitive closure (TC) \cite{immerman:book}, and TC is just the
computation of a connectivity relation in an implicit graph which
is easy to do in STRIPS.

\section{Aportes realizados}
\section{Direcciones futuras}
