% Chapter 5
\chapter{Conclusiones y Recomendaciones}
\label{Capitulo5}
\lhead{\emph{Conclusiones y Recomendaciones}}

Este trabajo ha presentado dos reducciones de la lógica de segundo orden a
problemas de planificación automática que se adhieren a la Teoría de
Complejidad Descriptiva. Las herramientas reciben como entrada una firma
$\sigma$, una sentencia en lógica de segundo orden (segundo orden existencial,
en el caso de la traducción de NP) y una estructura $\A\in\struc[\sigma]$, y
producen como salida un par de archivos en PDDL $\tup{\domain,\instance}$, que
representan un dominio y una instancia de planificación en STRIPS. La
conversión corre en tiempo polinomial en el tamaño de la estructura $\|\A\|$, y
por lo tanto puede usarse como un método eficiente para generar reducciones de
problemas NP a STRIPS. Esta salida puede utilizarse con un planificador 
para obtener una solución al problema original.

Los resultados de los experimentos muestran que la herramienta puede utilizarse
para crear solucionadores a problemas específicos NP o PH que tienen un buen desempeño
sobre problemas pequeños. Así, por ejemplo, un profesional del área de
computación que tenga la necesidad de resolver problemas pequeños de grafos, o
requiera realizar pruebas de concepto para estudiar problemas académicos o de la vida
real, puede seguir el siguiente proceso para conseguir una solución rápida que
cumpla con sus objetivos:
\begin{enumerate}
\item Comparar el problema de interés con problemas de decisión conocidos en el
área de teoría de grafos, optimización, u otras.
\item Modelar el problema de interés utilizando lógica de segundo orden, y
escribirlo en la sintaxis propuesta por este trabajo.
\item Utilizar la herramienta para producir el dominio y distintas instancias
en PDDL, usar un planificador para conseguir soluciones a estas instancias y
analizar las soluciones arrojadas para constatar cómo se interpretan en el contexto del
problema de interés.
\end{enumerate}

Debe notarse que no todos los problemas son fácilmente modelables utilizando un
lenguaje descriptivo como el que hemos desarrollado en este trabajo. Así como
no existe un lenguaje de programación \textbf{ideal} que permita la expresión
de cualquier problema de una forma sencilla, sino que existen distintos
\textbf{paradigmas} y el programador debe tener la experticia para reconocer
cuál es el más adecuado para cada problema determinado, no hay una única forma
``óptima'' de tratar con todos los problemas de decisión.
Por ejemplo, sería particularmente engorroso intentar
modelar el problema de \textit{subset sum} (\textbf{suma de subconjuntos}),
utilizando el lenguaje lógico descrito, a pesar de que es un problema
NP-completo, pues se debe simular un sistema formal
aritmético que permita la suma de números en tal lógica. La teoría establece
que es posible, pero no se recomienda el uso de la herramienta para ello ya que
el ejercicio puede tomar mucho tiempo.

Además, la herramienta es de interés para los investigadores del área de
planificación automática, pues los problemas de planificación que genera
presentan un reto para los planificadores actuales y pueden utilizarse como una
medida de comparación entre diferentes enfoques a cómo construir planificadores
generales y eficientes, particularmente en el caso de \textit{SAT-planners} y
en la búsqueda de planes paralelos.

La reducción a NP presentada en ese trabajo fue presentada en el mes de junio
de 2011 en la Conferencia Internacional de
Planificación Automática (ICAPS '11) en la ciudad de Freiburg, Alemania.
La extensión a la herramienta que le permite procesar problemas en PH ha sido
propuesta como artículo de investigación corto en ICAPS '13, conferencia que
tendrá lugar en Roma, Italia. Se recibirán noticias sobre su aceptación en
enero de 2013.

Para seguir adelante con esta línea de investigación se proponen las siguientes
recomendaciones:

\begin{enumerate}[--]
\item Diseñar una segunda extensión a la herramienta, modificando el lenguaje
para incorporar el formalismo de \textbf{clausura transitiva}, necesario para
resolver problemas generales de la clase de complejidad PSPACE. Con tal
extensión, el poder expresivo del lenguaje presentado aquí se equipararía a
PDDL.
\item Optimizar las reducciones para distintos tipos de planificadores, ya que
el trabajo se concentró en planfificadores basados en SAT. Como punto de
partida, puede investigarse a fondo por qué el planificador LAMA '11 tuvo
mejores resultados que M en el dominio \coCOL.
\end{enumerate}
