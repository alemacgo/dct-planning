\chapter{Experimentos y resultados}
\label{Chapter4}
\lhead{Capítulo 4. \emph{Experimentos y resultados}}

Este capítulo presenta una serie de experimentos realizados para evaluar hasta
qué punto es práctico resolver problemas NP y PH modelados en lógica y
traducidos por la herramienta.
En primer lugar, se justifica la escogencia de un planificador basado en SAT
para la realización de los experimentos y se presenta una forma de calcular
cotas inferiores y superiores para acotar la búsqueda de una solución. Luego,
se explica el protocolo de realización de los experimentos y se describen brevemente 
los problemas NP y PH que se utilizarán. Finalmente, se discuten los resultados
obtenidos en estos experimentos.

\section{Escogencia de planificador}
De acuerdo con \cite{russell:book}, las estrategias más comunes para resolver
un problema de planificación son la utilización de planificadores basados en
SAT (SAT-\textit{planners}), la búsqueda heurística y la búsqueda basada en un
grafo de planificación.

\cite{rintanen:notes} expone que para ciertos dominios es conveniente
considerar la noción de planes paralelos: planes que permiten la aplicación de
varias acciones ``a la vez'': si existen $n$ acciones que afectan y dependen de
condiciones disjuntas, hay $n!$ planes que son equivalentes (i.e., llevan al
mismo estado) que serían explorados por un planificador serial. Si $n$ es
grande, esto puede ser combinatoriamente complejo. Por tanto, un planificador
que tome en cuenta planes paralelos tendría ventajas sobre un planificador
serial.

Recuerde que en los dominios traducidos por la herramienta propuesta en este
trabajo, las acciones \texttt{colocar-verdadera}
aplicadas a diferentes variables o tuplas son completamente independientes, y
por tanto paralelizables.

Existen planificadores basados en SAT que resuelven problemas de
planificación paralela de manera muy eficiente. Se escogió el
SAT-\textit{planner} del
estado del arte \texttt{M} \citep{rintanen:m} para realizar los experimentos, 
debido a su superior desempeño.

\subsection{Ventanas de horizonte}

En esta sección se derivan cotas estrictas sobre la longitud de los planes
paralelos para los problemas resultantes. Estas cotas se utilizan con un
planificador basado en SAT para demostrar que un problema STRIPS no tiene
solución o para mejorar el desempeño de los planificadores.

Una ventana de horizonte para un problema STRIPS $P$ es un intervalo de la
forma $[i,f]$, tal que $P$ tiene solución si y sólo si tiene un plan paralelo 
de tamaño $\ell\in[i,f]$. Las ventanas se pueden utilizar para podar el espacio de
búsqueda.

La estructura recursiva del problema generado permite el cálculo de ventanas de
horizonte no triviales. Como todas las acciones \texttt{colocar-verdadera}
pueden ser aplicadas de manera concurrente, un plan paralelo necesita a lo sumo un paso para
ejecutarlas. El plan también requiere de las acciones \texttt{empezar-prueba}
y \texttt{probar-meta}. De este modo, la ventana de horizonte es $[2,3]$ más la
ventana de horizonte $\pwin(\psi)$ de la sentencia $\psi$. Las ventanas de
horizonte son definidas inductivamente por

\begin{enumerate}[--]
\item $\pwin(\theta)\doteq [0,0]$ si $\theta$ es un literal,
\item $\pwin(\wedge_{i=1}^n \theta_i)\doteq 1+\bigvee_{i=1}^n \pwin(\theta_i)$,
\item $\pwin(\vee_{i=1}^n \theta_i)\doteq 1+\bigwedge_{i=1}^n \pwin(\theta_i)$,
\item $\pwin((\exists y)\theta(\bar x,y))\doteq 1+\pwin(\theta)$, y
\item $\pwin((\forall y)\theta(\bar x,y))\doteq \|A\|+\pwin(\theta)$,
\end{enumerate}
donde $\A$ es la estructura asociada al problema, y las operaciones entre
ventanas y escalares son
$[a,b]\vee[a',b']\doteq[\max(a,a'),\max(b,b')]$,
$[a,b]\wedge[a',b']\doteq[\min(a,a'),\max(b,b')]$ y
$c+[a,b]\doteq[c+a,c+b]$.

SAT, por ejemplo, tiene la ventana $[\|\A\|+5,\|\A\|+6]$,
lo que significa que el CNF codificado por la estructura $\A$ es satisfacible
si y sólo si existe un plan paralelo de longitud $\|\A\|+5\leq\ell\leq\|\A\|+6$.

%By bounding the upper limit of parallel horizon 
%windows, we obtain the following surprising result.
%
%\begin{theorem}
%Consider a signature $\sigma$, $\Phi\in\SOE(\sigma)$ and
%$\A\in\struc[\sigma]$. Then, to decide $\A\models\Phi$,
%it is enough to consider parallel plans of makespan
%linear on $\|\A\|$ for fixed $\Phi$ but independently
%of the arities in $\sigma$ and $\Phi$.
%More precisely, it is enough to consider plans of makespan
%at most $q(\|\A\|-1)+d+3$ where $q$ is the maximum nesting
%of universal quantifiers in $\psi$, $d$ is the depth of
%$\psi$ and $\psi$ is the FOL part of $\Phi$.
%\end{theorem}
%\begin{proof}
%Let $n=\|\A\|$ and $T$ the parse tree of height $h$ for $\psi$.
%For a maximal branch $b\in T$, let $q_b$ be the number
%of universal quantifiers in $b$, $h_b$ its height, and
%$u(b)$ the upper limit of the parallel horizon window
%along $b$. The upper limit $u(\psi)$ of $\pwin(\psi)$
%is $\max_{b\in T} u(b)$. On the other hand,
%\[ u(b) = q_bn + h_b - q_b = q_b(n-1) + h_b \leq q(n-1) + h. \]
%End with $h=d$ and that $3$ must be added to $u(\psi)$.
%\end{proof}
%
%This bound is tight for SAT.
%The result is surprising because one would expect 
%the need to consider parallel plans of makespan $\O(\|\A\|^k)$
%for some $k$. However, note that a linear makespan
%does not mean a linear number of operators.

\section{Diseño de los experimentos}
%En esta sección se describe el protocolo experimental utilizado
%en este trabajo.

\subsection{Procedimiento}

Los experimentos consisten en la modelación, traducción y resolución
de problemas NP y PH. Para cada tipo de problema se ejecuta el planificador
\texttt{M} sobre instancias de
distintos tamaños y características, para luego analizar cuántos y cuáles
instancias fueron resueltas satisfactoriamente en un tiempo limitado, y cuáles
fueron los tiempos de corrida.
Los detalles sobre la modelación de cada problema pueden encontrarse en el
Apéndice \ref{apendiceA}.

\section{Experimentos y resultados de problemas NP}
Se realizaron experimentos en los problemas NP-completos: \SAT, \CLIQUE, \CHD
\ (Camino \textit{Hamiltoniano} Dirigido),
\TDM\ (\textit{3-Dimensional Matching}), \TCOL\ (3-colorabilidad) y \KCOL
\ ($k$-colorabilidad).
También se computó el número cromático de grafos aleatorios utilizando la
herramienta como un oráculo.

Los experimentos se realizaron en un procesador Intel Xeon corriendo a 1.86 GHz,
con 2 GB de memoria RAM. Cada instancia se intentó resolver utilizando el
planificador \texttt{M} durante 30 minutos, con un límite de 1GB de memoria.

A continuación se presentan los resultados desglosados por dominio.
Para cada tipo de problema, la tabla muestra el número de instancias resueltas
($N^*$), el número total de instancias ($N$), el número de instancias resueltas
que satisfacen la propiedad (\#pos), el número de instancias resueltas que
\textbf{no} satisfacen la propiedad (\#neg), y el tiempo promedio de resolución
en segundos. Sobre cada tabla se incluye el tamaño de la ventana de horizonte
particular del problema ($\pwin$).

\subsection{SAT}
Las instancias de SAT fueron tomadas del repositorio
SATLIB\footnote{\texttt{http://www.satlib.org}}, un sitio web hecho por
\cite{hoos:satlib} que aloja una librería de problemas SAT que pertenecen a la
región de la \textit{fase de transición} \citep{gent:transition}, es decir,
tienen ciertas propiedades que los hacen difíciles de resolver.
Los problemas de tipo uf20, uf50 y uf75 son instancias aleatorias con 20, 50 y
75 variables proposicionales, respectivamente, mientras que las instancias de
tipo uuf50 y uuf75 son no satisfacibles de tamaño 50 y 75.

\begin{table}[h!]
\begin{center}
\begin{tabular}{lllllll}
\multicolumn{5}{@{}c}{\footnotesize\SAT: $\pwin=[n+5,n+6]$} \\
\midrule
              &    $N^*$/$N$ & \#pos. & \#neg. & tiempo prom. \\
\midrule
%                                                             % sample standard deviation
uf20          &        40/40 &     40 &      0 &       1.7 \\ % 2.0
uf50          &        40/40 &     40 &      0 &     146.7 \\ % 202.3
uf75          &        15/40 &     15 &      0 &     362.1 \\ % 568.6
uuf50         &        40/40 &      0 &     40 &     548.5 \\ % 260.2
uuf75         &         1/40 &      0 &      1 &   1,746.4 \\ % 0.0
\midrule
\end{tabular}
\end{center}
\caption[Resultados de M para \SAT]{Resultados de M para \SAT}
%\label{table:results}
\end{table}

\subsection{Problemas de grafos y \TDM}
Las instancias de problemas de grafos se generaron de acuerdo con el modelo $G(n, p)$
\cite{bollobas:random-graphs}, variando $n$ y $p$, y las instancias para \TDM\ se generaron
escogiendo aleatoriamente tripletas de $\{0,\ldots,n-1\}^3$ con probabilidad
$p$, siendo $p$ variable. 

\begin{table}[h!]
\begin{center}
\begin{tabular}{lllllll}
\multicolumn{5}{@{}c}{\footnotesize\textsc{Clique}: $\pwin=[2n+4, 3n+7]$} \\
\midrule
              &    $N^*$/$N$ & \#pos. & \#neg. & tiempo prom. \\
\midrule
%5-3           &        40/40 &     10 &     30 &       0.0 \\ % 0.0
%5-4           &        40/40 &      0 &     40 &       0.2 \\ % 0.5
10-3          &        40/40 &     22 &     18 &       1.2 \\ % 0.8
10-4          &        40/40 &     12 &     28 &       2.2 \\ % 2.3
10-5          &        40/40 &      1 &     39 &      32.3 \\ % 106.7
15-3          &        40/40 &     22 &     18 &      10.5 \\ % 8.2
15-4          &        40/40 &     11 &     29 &      36.6 \\ % 84.5
15-5          &        39/40 &      4 &     35 &      74.3 \\ % 136.2
15-6          &        37/40 &      1 &     36 &      79.4 \\ % 128.3
20-3          &        40/40 &     25 &     15 &      40.2 \\ % 21.6
20-4          &        40/40 &     17 &     23 &      72.6 \\ % 66.3
20-5          &        39/40 &     10 &     29 &     159.6 \\ % 252.4
20-6          &        34/40 &      4 &     30 &     185.2 \\ % 225.2
25-3          &        40/40 &     30 &     10 &     111.9 \\ % 53.9
25-4          &        40/40 &     18 &     22 &     231.0 \\ % 236.7
25-5          &        39/40 &     10 &     29 &     387.5 \\ % 396.6
25-6          &        36/40 &      8 &     28 &     394.1 \\ % 321.1
\end{tabular}
\end{center}
\caption[Resultados de M para \CLIQUE]{Resultados de M para \CLIQUE. Se utiliza
la notación $x-y$ para denotar que el problema fue hallar una clique de tamaño
$y$ en un grafo con $x$ nodos.}
%\label{table:results}
\end{table}

\begin{table}[h!]
\begin{center}
\begin{tabular}{lllllll}
\multicolumn{5}{@{}c}{\footnotesize\CHD: $\pwin=[n+3,n+10]$} \\
\midrule
              &    $N^*$/$N$ & \#pos. & \#neg. & tiempo prom. \\
\midrule
%5             &        40/40 &     12 &     28 &       0.0 \\ % 0.0
10            &        40/40 &     15 &     25 &       1.1 \\ % 2.3
15            &        39/40 &     18 &     21 &      63.7 \\ % 203.3
20            &        31/40 &     20 &     11 &      70.0 \\ % 127.6
25            &        29/40 &     20 &      9 &     202.1 \\ % 199.8
30            &        22/40 &     20 &      2 &     629.1 \\ % 242.2
\midrule
\end{tabular}
\end{center}
\caption[Resultados de M para \CHD]{Resultados de M para \CHD. La primera
columna indica el número de nodos del grafo.}
%\label{table:results}
\end{table}

\begin{table}[h!]
\begin{center}
\begin{tabular}{lllllll}
\multicolumn{5}{@{}c}{\footnotesize\TDM: $\pwin=[3n+4,3n+6]$} \\
\midrule
              &    $N^*$/$N$ & \#pos. & \#neg. & tiempo prom. \\
\midrule
%5             &        40/40 &     19 &     21 &       0.0 \\ % 0.0
10            &        40/40 &     36 &      4 &       9.6 \\ % 2.8
15            &        40/40 &     40 &      0 &     251.5 \\ % 65.5
20            &        13/40 &     13 &      0 &   1,191.0 \\ % 42.1 (avg/std) is over 3 instances !!!
25            &         0/40 &      0 &      0 &       --- \\ % ---
\midrule
\end{tabular}
\end{center}
\caption[Resultados de M para \TDM]{Resultados de M para \TDM. La primera
columna indica la cardinalidad del conjunto.}
%\label{table:results}
\end{table}

\begin{table}[h!]
\begin{center}
\begin{tabular}{lllllll}
\multicolumn{5}{@{}c}{\footnotesize\TCOL: $\pwin=[2n+4, 2n+7]$} \\
\midrule
              &    $N^*$/$N$ & \#pos. & \#neg. & tiempo prom. \\
\midrule
%5             &        40/40 &     37 &      3 &       0.0 \\ % 0.0
10            &        40/40 &     18 &     22 &       0.1 \\ % 0.1
15            &        40/40 &     24 &     16 &       0.9 \\ % 0.6
20            &        40/40 &     12 &     28 &       3.0 \\ % 1.8
25            &        40/40 &     30 &     10 &       8.9 \\ % 4.7
30            &        40/40 &      9 &     31 &      20.9 \\ % 12.2
40            &        40/40 &      4 &     36 &      75.1 \\ % 55.7
50            &        40/40 &      1 &     39 &     196.7 \\ % 119.2
\midrule
\end{tabular}
\end{center}
\caption[Resultados de M para \TCOL]{Resultados de M para \TCOL. La primera
columna indica el número de nodos del grafo.}
%\label{table:results}
\end{table}

\begin{table}[h!]
\begin{center}
\begin{tabular}{lllllll}
\multicolumn{5}{@{}c}{\footnotesize\KCOL: $\pwin=[2n+4,3n+6]$} \\
\midrule
              &    $N^*$/$N$ & \#pos. & \#neg. & tiempo prom. \\
\midrule
%5-2           &        40/40 &     20 &     20 &       0.0 \\ % 0.0
%5-3           &        40/40 &     37 &      3 &       0.0 \\ % 0.0
%5-4           &        40/40 &     39 &      1 &       0.0 \\ % 0.0
10-2          &        40/40 &      9 &     31 &       1.9 \\ % 0.5
10-3          &        40/40 &     18 &     22 &       2.8 \\ % 1.4
10-4          &        40/40 &     27 &     13 &      11.0 \\ % 23.3
15-2          &        40/40 &      7 &     33 &      33.5 \\ % 9.2
15-3          &        40/40 &     16 &     24 &      46.5 \\ % 12.3
15-4          &        40/40 &     24 &     16 &      91.7 \\ % 113.5
20-2          &        40/40 &      3 &     37 &     254.9 \\ % 68.2
20-3          &        40/40 &     12 &     28 &     395.9 \\ % 221.8
20-4          &        40/40 &     20 &     20 &     497.3 \\ % 178.0
25-2          &         0/40 &      0 &      0 &       --- \\ % ---
25-3          &         0/40 &      0 &      0 &       --- \\ % ---
25-4          &         0/40 &      0 &      0 &       --- \\ % ---
\midrule
\end{tabular}
\end{center}
\caption[Resultados de M para \KCOL]{Resultados de M para \KCOL. La primera
columna indica el número de nodos del grafo y el número de colores de los que
se intenta colorear.}
%\label{table:results}
\end{table}

\begin{table}[h!]
\begin{center}
\begin{tabular}{lllllll}
              &    $N^*$/$N$ & \#pos. & \#neg. & tiempo prom. \\
\midrule
Total         &  1,614/1,920 &    706 &    908 &     180.9 \\ % a rellenar al final
\end{tabular}
\end{center}
\caption[Resultados de M para *]{Resultados de M para *}
%\label{table:results}
\end{table}

\begin{table}[h!]
\centering
\small
\begin{tabular}{lllllllll}
       &       & \multicolumn{7}{c}{$k$-colorabilidad} \\
\cmidrule(l){3-9}
instancia      & $\chi$ &     1 &     2 &     3 &     4 &     5 &     6 &     7 \\
\midrule
10-0.75 (1)      &      5 &     2 &     2 &     6 &   101 &\bf  3 &       &       \\
10-0.75 (2)      &      5 &     1 &     2 &     2 &     6 &\bf  4 &       &       \\
10-0.85        &      7 &     2 &     2 &     3 &     6 &     4 & 1,265 &\bf  4 \\
15-0.25        &      2 &    27 &\bf 62 &       &       &       &       &       \\
15-0.60        &      5 &    27 &    29 &    54 &   118 &\bf 72 &       &       \\
15-0.70        &      6 &    28 &    28 &    33 &    47 &   329 &\bf 67 &       \\
20-0.10        &      3 &   214 &   350 &\bf705 &       &       &       &       \\ % last number strange
20-0.25        &      4 &   211 &   272 & 1,261 &\bf837 &       &       &       \\
\midrule
\end{tabular}
\caption[Resultados de M sobre números cromáticos]{
Resultados de M sobre el cómputo de números cromáticos en grafos aleatorios.
La primera columna muestra el número de nodos del grafo y su probabilidad según
el modelo $G(n,p)$.
Para cada instancia, la tabla muestra el número cromático $\chi$, el tiempo
(en segundos) para demostrar o descartar la $k$-colorabilidad para valores
incrementales de $k$.
El último valor de $k$ para cada instancia es el número cromático.}
\label{table:chromatic}
\end{table}

The chromatic number of a graph $G=(V,E)$ is the least
$k$ such that $G$ is $k$-colorable. It is NP-hard to 
compute the chromatic number of a graph, but we can
do it by testing for $k$-colorability for increasing
values of $k=1,\ldots,|V|$.\footnote{One can do better
by performing a binary search on $k$.}
Table~\ref{table:chromatic} shows results for the computation
of chromatic numbers on random graphs.
For each instance, it shows the chromatic number $\chi$
and the time to prove/disprove the existence of a
$k$-coloring for increasing values of $k$.

\section{Resultados en resolución de problemas PH}
\begin{table}[t]
\centering
\resizebox{3.3in}{!}{
  \begin{tabular}{lllrrrrrrrrrl}
    %\toprule
    \qEA  & $\#\exists$ & $\#\forall$ &             & $n$ & $+$ & $-$ &    time &   len & {\sc\small pddl} \\
    \midrule
          &          60 &           1 &             &   5 & --- &   5 &   184.3 &    na &             18.4 \\
          &             &           2 &             &   5 & --- &   1 & 4,382.5 &    na &             18.5 \\
    \cmidrule{2-10}
          &         100 &           1 &             &   5 &   4 &   1 &   176.6 &   316 &             20.4 \\ % #actions=36,017
          &             &           2 &             &   5 &   3 &   2 & 3,471.9 &   628 &             20.5 \\ % #actions= 2,737
    \midrule
    \qEAE & $\#\exists$ & $\#\forall$ & $\#\exists$ & $n$ & $+$ & $-$ &    time &   len & {\sc\small pddl} \\
    \midrule
          &          10 &           2 &          30 &   5 & --- &   5 & 4,199.2 &    na &             17.5 \\
          &             &           2 &          50 &   5 & --- &   5 & 2,313.9 &    na &             18.4 \\
    \cmidrule{2-10}
          &          30 &           2 &          30 &   5 & --- &   5 & 3,210.7 &    na &             18.5 \\
          &             &           2 &          50 &   5 & --- &   5 & 3,166.3 &    na &             19.4 \\
    \cmidrule{2-10}
          &          50 &           2 &          30 &   5 & --- &   1 & 3,313.4 &    na &             19.4 \\
          &             &           2 &          50 &   5 &   3 &   2 & 3,450.9 &   640 &             20.4 \\ % #actions=3,182.0
    \midrule
    \qAE  & $\#\forall$ & $\#\exists$ &             & $n$ & $+$ & $-$ &    time &   len & {\sc\small pddl} \\
    \midrule
          &           1 &         100 &             &   5 &   5 & --- &   147.9 &   319 &             20.2 \\ % #actions=45,378.6
    \cmidrule{2-10}
          &           2 &          30 &             &   5 & --- &   5 & 2,060.4 &    na &             16.9 \\
          &             &          60 &             &   5 &   4 &   1 & 3,100.7 &   637 &             18.4 \\ % #actions=3,086.0
          &             &          80 &             &   5 &   5 & --- & 2,893.2 &   637 &             19.3 \\ % #actions=2,961.4
          &             &         100 &             &   5 &   5 & --- & 2,092.8 &   637 &             20.2 \\ % #actions=3,171.8
    \cmidrule{2-10}
          &           3 &          15 &             &   5 & --- & --- &      na &    na &              --- \\
    \midrule
    \qAEA & $\#\forall$ & $\#\exists$ & $\#\forall$ & $n$ & $+$ & $-$ &    time &   len & {\sc\small pddl} \\
    \midrule
          &           1 &          60 &           1 &   5 &   1 & --- &   404.7 &   635 &             21.1 \\
          &             &          60 &           2 &   5 & --- & --- &     --- &    na &             21.2 \\
          &             &          80 &           1 &   5 &   5 & --- &   403.2 &   635 &             22.0 \\
          &             &         100 &           1 &   5 &   5 & --- &   390.8 &   635 &             23.1 \\
    \cmidrule{2-10}
          &           2 &          60 &           1 &   5 &   2 & --- &   456.7 & 1,269 &             21.2 \\
          &             &          60 &           2 &   5 & --- & --- &     --- &    na &             21.3 \\
          &             &          80 &           1 &   5 &   5 & --- &   499.5 & 1,269 &             22.1 \\
          &             &         100 &           1 &   5 &   5 & --- &   454.9 & 1,269 &             23.2 \\
    %\bottomrule
  \end{tabular}
}
\caption{\small Random QBF problems with 150 clauses each
  and 4 types of QBFs: \qEA, \qEAE, \qAE and \qAEA.
  Columns for number of variables of each type, number of instances ($n$),
  number of positive ($+$) and negative ($-$) instances, and averages for
  time (in seconds), \emph{parallel} plan length (for solved instances),
  and PDDL size (in KB).
}
\label{table:exp:qbf}
\end{table}

\begin{table}[h!]
\centering
\begin{tabular}{crrrrrr}
$|V|$ & $n$ & $+$ & $-$ &  time & plan len & {\sc\small pddl} \\
\midrule
    4 &   5 &   1 &   4 &   0.6 &   1,731 & 0.4 \\
    5 &   5 &   2 &   3 &  41.0 &   6,695 & 0.6 \\
    6 &   5 &   2 &   3 & 280.3 &  26,163 & 0.7 \\
    7 &   5 &   2 &   2 &  38.2 & 102,935 & 0.8 \\
    8 &   5 &   1 &   2 & 211.9 & 406,851 & 1.0 \\
    9 &   5 & --- &   1 &   0.3 &      na & 1.1 \\
\end{tabular}
\caption{\small \coCOL on random graphs.
  Columns for number of vertices ($|V|$), number of instances ($n$),
  number of positive ($+$) and negative ($-$) instances, and averages for
  time (in seconds), plan length (for solved positive instances), and
  PDDL size (in KB).
}
\label{table:exp:co-3col}
\end{table}
