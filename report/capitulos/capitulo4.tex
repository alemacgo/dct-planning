\chapter{Experimentos y resultados}
\label{Chapter4}
\lhead{Capítulo 4. \emph{Experimentos y resultados}}

Este capítulo presenta una serie de experimentos realizados para evaluar hasta
qué punto es práctico resolver problemas NP y PH modelados en lógica y
traducidos por la herramienta.
En primer lugar, se justifica la escogencia de un planificador basado en SAT
para la realización de los experimentos y se presenta una forma de calcular
cotas inferiores y superiores para acotar la búsqueda de una solución. Luego,
se explica el protocolo de realización de los experimentos y se describen brevemente 
los problemas NP y PH que se utilizarán. Finalmente, se discuten los resultados
obtenidos en estos experimentos.

\section{Escogencia de planificador}
De acuerdo con \cite{russell:book}, las estrategias más comunes para resolver
un problema de planificación son la utilización de planificadores basados en
SAT (SAT-\textit{planners}), la búsqueda heurística y la búsqueda basada en un
grafo de planificación.

\cite{rintanen:notes} expone que para ciertos dominios es conveniente
considerar la noción de planes paralelos: planes que permiten la aplicación de
varias acciones ``a la vez'': si existen $n$ acciones que afectan y dependen de
condiciones disjuntas, hay $n!$ planes que son equivalentes (i.e., llevan al
mismo estado) que serían explorados por un planificador serial. Si $n$ es
grande, esto puede ser combinatoriamente complejo. Por tanto, un planificador
que tome en cuenta planes paralelos tendría ventajas sobre un planificador
serial.

Recuerde que en los dominios traducidos por la herramienta propuesta en este
trabajo, las acciones \texttt{colocar\_verdadera} y \texttt{colocar\_falsa}
aplicadas a diferentes variables o tuplas son completamente independientes, y
por tanto paralelizables.

Existen SAT-\textit{planners} que resuelven problemas de
planificación paralela de manera muy eficiente. Se escogió el planificador del
estado del arte basado en SAT \texttt{M} \citep{rintanen:m} para realizar los experimentos, 
debido a su superior desempeño.

\subsection{Ventanas de horizonte}

In the following, we derive tight bounds on the length
of parallel plans for the resulting problems. These bounds
are used with SAT-based planner to show that a given
STRIPS problem has no solution and also to improve the
performance of the planners.

A horizon window for a STRIPS problem $P$ is an
interval of the form $[s,f]$ such that $P$ has a
plan iff it has a plan of length $\ell\in[s,f]$.
A window is a parallel-horizon window if $\ell$
refers to the makespan of a parallel plan.
Horizon windows can be effectively used to prune
the search space.

The recursive structure of the generated problem
permits the calculation of non-trivial horizon
windows and of tight parallel-horizon windows.
Indeed, since all set operators can be applied
concurrently, a parallel plan needs at most one
time step to execute them. The plan also requires
the operators \fttt{begin-proof} and \fttt{prove-goal}.
Thus, the parallel-horizon window is $[2,3]$ (the lower
bound 2 applies when there is a plan that uses no
set operators) plus the parallel-horizon window
$\pwin(\psi)$ of the sentence $\psi$;
parallel-horizon windows are inductively defined by
\begin{enumerate}[--]
\item $\pwin(\theta)\doteq [0,0]$ if $\theta$ is a literal,
\item $\pwin(\wedge_{i=1}^n \theta_i)\doteq 1+\bigvee_{i=1}^n \pwin(\theta_i)$,
\item $\pwin(\vee_{i=1}^n \theta_i)\doteq 1+\bigwedge_{i=1}^n \pwin(\theta_i)$,
\item $\pwin((\exists y)\theta(\bar x,y))\doteq 1+\pwin(\theta)$, and
\item $\pwin((\forall y)\theta(\bar x,y))\doteq \|A\|+\pwin(\theta)$,
\end{enumerate}
where $\A$ is the structure associated to the problem,
and the operations between windows and scalars are
$[a,b]\vee[a',b']\doteq[\max\{a,a'\},\max\{b,b'\}]$,
$[a,b]\wedge[a',b']\doteq[\min\{a,a'\},\max\{b,b'\}]$ and
$c+[a,b]\doteq[c+a,c+b]$.
SAT, for example, has the window $[\|\A\|+5,\|\A\|+6]$
which means that the CNF encoded by a structure $\A$
is satisfiable iff there is a parallel plan of
makespan $\ell$ such that $\|\A\|+5\leq\ell\leq\|\A\|+6$.

By bounding the upper limit of parallel horizon 
windows, we obtain the following surprising result.

\begin{theorem}
Consider a signature $\sigma$, $\Phi\in\SOE(\sigma)$ and
$\A\in\struc[\sigma]$. Then, to decide $\A\models\Phi$,
it is enough to consider parallel plans of makespan
linear on $\|\A\|$ for fixed $\Phi$ but independently
of the arities in $\sigma$ and $\Phi$.
More precisely, it is enough to consider plans of makespan
at most $q(\|\A\|-1)+d+3$ where $q$ is the maximum nesting
of universal quantifiers in $\psi$, $d$ is the depth of
$\psi$ and $\psi$ is the FOL part of $\Phi$.
\end{theorem}
\begin{proof}
Let $n=\|\A\|$ and $T$ the parse tree of height $h$ for $\psi$.
For a maximal branch $b\in T$, let $q_b$ be the number
of universal quantifiers in $b$, $h_b$ its height, and
$u(b)$ the upper limit of the parallel horizon window
along $b$. The upper limit $u(\psi)$ of $\pwin(\psi)$
is $\max_{b\in T} u(b)$. On the other hand,
\[ u(b) = q_bn + h_b - q_b = q_b(n-1) + h_b \leq q(n-1) + h. \]
End with $h=d$ and that $3$ must be added to $u(\psi)$.
\end{proof}

This bound is tight for SAT.
The result is surprising because one would expect 
the need to consider parallel plans of makespan $\O(\|\A\|^k)$
for some $k$. However, note that a linear makespan
does not mean a linear number of operators.

\section{Diseño de los experimentos}
\subsection{Procedimiento}

\subsection{Modelaje de problemas NP}

\subsection{Modelaje de problemas PH}

\section{Resultados en resolución de problemas NP}

\section{Resultados en resolución de problemas PH}

Lorem Ipsum Blah Lorem Ipsum Blah Lorem Ipsum Blah Lorem Ipsum Blah Lorem Ipsum Blah
Lorem Ipsum Blah Lorem Ipsum Blah Lorem Ipsum Blah Lorem Ipsum Blah Lorem Ipsum Blah
Lorem Ipsum Blah Lorem Ipsum Blah Lorem Ipsum Blah Lorem Ipsum Blah Lorem Ipsum Blah
Lorem Ipsum Blah Lorem Ipsum Blah Lorem Ipsum Blah Lorem Ipsum Blah Lorem Ipsum Blah
Lorem Ipsum Blah Lorem Ipsum Blah Lorem Ipsum Blah Lorem Ipsum Blah Lorem Ipsum Blah
Lorem Ipsum Blah Lorem Ipsum Blah Lorem Ipsum Blah Lorem Ipsum Blah Lorem Ipsum Blah
\begin{table}[t]
\begin{center}
\begin{tabular}{lllllll}
\multicolumn{5}{@{}c}{\footnotesize\textsc{Sat}: $\pwin=[n+5,n+6]$} \\
\midrule
              &    $N^*$/$N$ & \#pos. & \#neg. & avg.\ time \\
\midrule
%                                                             % sample standard deviation
uf20          &        40/40 &     40 &      0 &       1.7 \\ % 2.0
uf50          &        40/40 &     40 &      0 &     146.7 \\ % 202.3
uf75          &        15/40 &     15 &      0 &     362.1 \\ % 568.6
uuf50         &        40/40 &      0 &     40 &     548.5 \\ % 260.2
uuf75         &         1/40 &      0 &      1 &   1,746.4 \\ % 0.0
\midrule
\end{tabular}
\end{center}
\caption[Resultados de M para *]{Resultados de M para *}
%\label{table:results}
\end{table}

Lorem Ipsum Blah Lorem Ipsum Blah Lorem Ipsum Blah Lorem Ipsum Blah Lorem Ipsum Blah
Lorem Ipsum Blah Lorem Ipsum Blah Lorem Ipsum Blah Lorem Ipsum Blah Lorem Ipsum Blah
Lorem Ipsum Blah Lorem Ipsum Blah Lorem Ipsum Blah Lorem Ipsum Blah Lorem Ipsum Blah
Lorem Ipsum Blah Lorem Ipsum Blah Lorem Ipsum Blah Lorem Ipsum Blah Lorem Ipsum Blah
Lorem Ipsum Blah Lorem Ipsum Blah Lorem Ipsum Blah Lorem Ipsum Blah Lorem Ipsum Blah
Lorem Ipsum Blah Lorem Ipsum Blah Lorem Ipsum Blah Lorem Ipsum Blah Lorem Ipsum Blah
\begin{table}[t]
\begin{center}
\begin{tabular}{lllllll}
\multicolumn{5}{@{}l}{\footnotesize\textsc{Clique}: $\pwin=[2n+4, 3n+7]$} \\
\midrule
              &    $N^*$/$N$ & \#pos. & \#neg. & avg.\ time \\
\midrule
%5-3           &        40/40 &     10 &     30 &       0.0 \\ % 0.0
%5-4           &        40/40 &      0 &     40 &       0.2 \\ % 0.5
10-3          &        40/40 &     22 &     18 &       1.2 \\ % 0.8
10-4          &        40/40 &     12 &     28 &       2.2 \\ % 2.3
10-5          &        40/40 &      1 &     39 &      32.3 \\ % 106.7
15-3          &        40/40 &     22 &     18 &      10.5 \\ % 8.2
15-4          &        40/40 &     11 &     29 &      36.6 \\ % 84.5
15-5          &        39/40 &      4 &     35 &      74.3 \\ % 136.2
15-6          &        37/40 &      1 &     36 &      79.4 \\ % 128.3
20-3          &        40/40 &     25 &     15 &      40.2 \\ % 21.6
20-4          &        40/40 &     17 &     23 &      72.6 \\ % 66.3
20-5          &        39/40 &     10 &     29 &     159.6 \\ % 252.4
20-6          &        34/40 &      4 &     30 &     185.2 \\ % 225.2
25-3          &        40/40 &     30 &     10 &     111.9 \\ % 53.9
25-4          &        40/40 &     18 &     22 &     231.0 \\ % 236.7
25-5          &        39/40 &     10 &     29 &     387.5 \\ % 396.6
25-6          &        36/40 &      8 &     28 &     394.1 \\ % 321.1
\end{tabular}
\end{center}
\caption[Resultados de M para *]{Resultados de M para *}
%\label{table:results}
\end{table}

\begin{table}[t]
\begin{center}
\begin{tabular}{lllllll}
\multicolumn{5}{@{}l}{\footnotesize\textsc{DirectedHamiltonianPath}: $\pwin=[n+3,n+10]$} \\
\midrule
              &    $N^*$/$N$ & \#pos. & \#neg. & avg.\ time \\
\midrule
%5             &        40/40 &     12 &     28 &       0.0 \\ % 0.0
10            &        40/40 &     15 &     25 &       1.1 \\ % 2.3
15            &        39/40 &     18 &     21 &      63.7 \\ % 203.3
20            &        31/40 &     20 &     11 &      70.0 \\ % 127.6
25            &        29/40 &     20 &      9 &     202.1 \\ % 199.8
30            &        22/40 &     20 &      2 &     629.1 \\ % 242.2
\midrule
\end{tabular}
\end{center}
\caption[Resultados de M para *]{Resultados de M para *}
%\label{table:results}
\end{table}

\begin{table}[t]
\begin{center}
\begin{tabular}{lllllll}
\multicolumn{5}{@{}l}{\footnotesize\textsc{3-DimensionalMatching}: $\pwin=[3n+4,3n+6]$} \\
\midrule
              &    $N^*$/$N$ & \#pos. & \#neg. & avg.\ time \\
\midrule
%5             &        40/40 &     19 &     21 &       0.0 \\ % 0.0
10            &        40/40 &     36 &      4 &       9.6 \\ % 2.8
15            &        40/40 &     40 &      0 &     251.5 \\ % 65.5
20            &        13/40 &     13 &      0 &   1,191.0 \\ % 42.1 (avg/std) is over 3 instances !!!
25            &         0/40 &      0 &      0 &       --- \\ % ---
\midrule
\end{tabular}
\end{center}
\caption[Resultados de M para *]{Resultados de M para *}
%\label{table:results}
\end{table}

\begin{table}[t]
\begin{center}
\begin{tabular}{lllllll}
\multicolumn{5}{@{}l}{\footnotesize\textsc{3-Colorability}: $\pwin=[2n+4, 2n+7]$} \\
\midrule
              &    $N^*$/$N$ & \#pos. & \#neg. & avg.\ time \\
\midrule
%5             &        40/40 &     37 &      3 &       0.0 \\ % 0.0
10            &        40/40 &     18 &     22 &       0.1 \\ % 0.1
15            &        40/40 &     24 &     16 &       0.9 \\ % 0.6
20            &        40/40 &     12 &     28 &       3.0 \\ % 1.8
25            &        40/40 &     30 &     10 &       8.9 \\ % 4.7
30            &        40/40 &      9 &     31 &      20.9 \\ % 12.2
40            &        40/40 &      4 &     36 &      75.1 \\ % 55.7
50            &        40/40 &      1 &     39 &     196.7 \\ % 119.2
\midrule
\end{tabular}
\end{center}
\caption[Resultados de M para *]{Resultados de M para *}
%\label{table:results}
\end{table}

\begin{table}[t]
\begin{center}
\begin{tabular}{lllllll}
\multicolumn{5}{@{}l}{\footnotesize$k$-\textsc{Colorability}: $\pwin=[2n+4,3n+6]$} \\
\midrule
              &    $N^*$/$N$ & \#pos. & \#neg. & avg.\ time \\
\midrule
%5-2           &        40/40 &     20 &     20 &       0.0 \\ % 0.0
%5-3           &        40/40 &     37 &      3 &       0.0 \\ % 0.0
%5-4           &        40/40 &     39 &      1 &       0.0 \\ % 0.0
10-2          &        40/40 &      9 &     31 &       1.9 \\ % 0.5
10-3          &        40/40 &     18 &     22 &       2.8 \\ % 1.4
10-4          &        40/40 &     27 &     13 &      11.0 \\ % 23.3
15-2          &        40/40 &      7 &     33 &      33.5 \\ % 9.2
15-3          &        40/40 &     16 &     24 &      46.5 \\ % 12.3
15-4          &        40/40 &     24 &     16 &      91.7 \\ % 113.5
20-2          &        40/40 &      3 &     37 &     254.9 \\ % 68.2
20-3          &        40/40 &     12 &     28 &     395.9 \\ % 221.8
20-4          &        40/40 &     20 &     20 &     497.3 \\ % 178.0
25-2          &         0/40 &      0 &      0 &       --- \\ % ---
25-3          &         0/40 &      0 &      0 &       --- \\ % ---
25-4          &         0/40 &      0 &      0 &       --- \\ % ---
\midrule
\end{tabular}
\end{center}
\caption[Resultados de M para *]{Resultados de M para *}
%\label{table:results}
\end{table}

\begin{table}[t]
\begin{center}
\begin{tabular}{lllllll}
              &    $N^*$/$N$ & \#pos. & \#neg. & avg.\ time \\
\midrule
Total         &  1,614/1,920 &    706 &    908 &     180.9 \\ % a rellenar al final
\end{tabular}
\end{center}
\caption[Resultados de M para *]{Resultados de M para *}
%\label{table:results}
\end{table}

\onecolumn
