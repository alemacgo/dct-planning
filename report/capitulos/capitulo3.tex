% Chapter 4

\chapter{Traducción de problemas PH}
\label{Chapter3}
\lhead{Capítulo 3. \emph{Traducción de problemas PH}}

In a recent contribution,
(hereafter referred to as PMB) show how to automatically reduce
instances of decision problems in NP into plan-existence decision
problems for \STRIPS.
In this approach, a decision problem $\Pi$ in NP is encoded as a
second-order existential (\SOE) formula $\Phi$ while a particular
instance is encoded as first-order structure $\A$ in a way that
$\A\models\Phi$ iff the instance encoded by $\A$ satisfies belongs to $\Pi$.
This is a general and feasible idea as it is known that \SOE
captures the class NP \cite{immerman:book}.
The pair $\tup{\Phi,\A}$ is then fed into a (software) tool that
outputs a \STRIPS problem $P$ that has a solution (plan) iff $\A\models\Phi$,
thus allowing the utilization of the current planning technology
to automatically solve the problems in NP that are expressed as
second-order formulas.
Moreover, the problem $P$ is not an arbitrary \STRIPS problem but
one that can be solved in polytime by a non-deterministic
Turing machine, thus guaranteeing that the whole approach is not
an overkill from the standpoint of complexity theory.
Furthermore, the function $\tup{\Phi,\cdot}:\struc\mapsto\STRIPS$ 
that is implemented by the tool for fixed $\Phi$ and that maps
first-order structures $\A$ into \STRIPS problems $P$ is computable
in polynomial time and thus corresponds to a genuine \emph{polytime many-one reduction}
from $\Pi$ into \STRIPS.

Since it is known that other (bigger) complexity classes are also
captured in second-order logic, PMB left as an open issue the
development of a tool capable of targeting such classes.
This paper bridges this gap and shows in a concise and crisp manner
how the tool can be extended to work for decision problems in the
Polynomial-time Hierarchy (\PH) \cite{sipser:book}.
Yet, although the resulting reductions will run in polynomial
time, the resulting \STRIPS problems will not be solvable in
non-deterministic polynomial time.
This is so because such problems only admit plans of
\emph{exponential length} in the worst case as \PH contains, besides
NP, the classes coNP and $\Sigma^p_k$ for any $k\geq 1$.
The resulting planning problems are thus short, meaning that they can
be encoded by short bit sequences (relative to the encoding length of
$\A$) but have long solutions.

The consequences of having small problems with long solutions
are several. Among others, such problems pose a difficult
challenge on the current state-of-the-art planners which are
all based on heuristic search. In particular, heuristics functions
that are based on the delete relaxation or that do not take into
account that the same action may be applied a large (i.e., exponential)
number of times will probably fail at being effective for guiding
the search.
Surprisingly, the current best SAT-based planner, the planner M,
is able to solve some of these challenging
\STRIPS problems for which forward-search planners like LAMA'11
are totally lost. Yet, more surprisingly, LAMA'11 is quite
superior to M in one of the tested domains (more about this below).

Being this a short paper, we don't have the space to present
a detailed account of the approach; the reader is referred
to PMB's work for it.
In the following sections, we revise some known results from
Descriptive Complexity, describe how problems in PH can be
automatically translated into \STRIPS, and present experimental
results.


\section{Capturing \PH in Second-Order Logic}

NP is captured by \SOE \cite{fagin:spectra}.
This result means that any decision problem $\Pi$ in NP can be
characterized by a second-order formula $\Phi$ that looks like
\begin{equation}
\label{eq:soe}
\Phi = (\exists R_1^{a_1})\cdots(\exists R_n^{a_n})\psi
\end{equation}
where each symbol $R_i^{a_i}$ is a existentially-quantified relation
of arity $a_i$, and $\psi$ is an arbitrary first-order sentence 
over a vocabulary containing $\{R_1^{a_1},\ldots,R_n^{a_n}\}$.
That is, $\Pi$ is the set of instances (words in a language)
that when encoded as structures $\A$ satisfy the formula $\Phi$.

Likewise, coNP is captured by \SOA that contains formulas
like Eq.~\eqref{eq:soe} but with the second-order existential
quantifiers replaced by second-order universal quantifiers.
In general, $\Sigma^p_k$ is captured by formulas
whose second-order quantifiers display a structure of $k$
alternating blocks of quantifiers, beginning with existentials.
For example, $\Sigma^p_2=\SOEA$ corresponds to formulas of
the form:
\begin{equation}
\Phi = (\exists R_1^{a_1}) \cdots (\exists R_n^{a_n}) 
       (\forall S_1^{a'_1}) \cdots (\forall S_m^{a'_m})
       \psi \,.
\end{equation}
Since \PH equals $\bigcup_{k\geq 1}\Sigma^p_k$ and is
fully characterized by arbitrary second-order formulas \cite{immerman:book},
the extended tool we sought must target such formulas
instead of the more restricted \SOE formulas.
On the other hand, we know that $\PH\subseteq\PSPACE$, the
decision problem for \STRIPS is \PSPACE-complete and there
are no known syntactic restrictions on \STRIPS that place
its decision problem in \PH. Thus, the new tool will focus
on generating (unrestricted) \STRIPS problems.

Consider now the simple formula $\Phi=(\forall R^1)\psi$
and an structure $\A$ with universe $|\A|$ (i.e., $|\A|$
is the set of objects in $\A$).
This formula is valid in $\A$ iff for \emph{every interpretation}
$R^\A$ of $R$, where $R^\A\subseteq|\A|$, we have $\tup{\A,R^\A}\models\psi$.
Thus, since there are $2^{\|\A\|}$ different interpretations for $R$,
a proof for $\A\models\Phi$ may be of \emph{exponential length}.
For example, if $\Phi_{\UNSAT}$ denotes UNSAT, then a proof that a
structure $\A$ (encoding a set of clauses) is unsatisfiable is
typically of exponential length as it must consider all the truth
valuations for the propositional variables in $\A$.
Indeed, the solutions for the planning problem $P$ generated
by the pair $\tup{\Phi_\UNSAT,\A}$ encode such proofs and are
of exponential length.


\section{Automatic Translations into \STRIPS}

Before extending PMB's work over \PH, we found convenient to add
to it a simple \emph{type system}. We refer to a
system $\t^*$ made from a finite set $\t=\{\mathbf{t}_0,\mathbf{t}_1,\ldots,\mathbf{t}_N\}$
of \emph{atomic types} where $\t^*$ is the smallest set that
contains $\t$ and all types of the form $t=t'\times t''$
for $t'\in\t^*$ and $t''\in\t$.
The idea is that each object in a first-order structure
is assigned an atomic type, with $\mathbf{t}_0$ standing for the
type containing all objects.
The types are used to qualify all the first-order and
second-order quantifications in formulas.
For example, $(\forall x\in t)$ refers to a first-order
quantification over all objects of type $t\in\t$, while
$(\forall R^t)$ refers to a second-order universally
quantified predicate $R$ with type $t\in\t^*$.
The only restriction that we pose is that the complex
types in $\t^*\setminus\t$ should only appear in
second-order quantifications.

\medskip

Let us consider a general \SO formula of the form 
\begin{equation}
\Phi = (\Q_1R_1^{t_1})(\Q_2R_2^{t_2})\cdots(\Q_nR_n^{t_n})\psi
\end{equation}
where each $\Q_i\in\{\exists,\forall\}$ is a second-order
quantifier,  $R_i$ is a relational symbol of type $t_i\in\t^*$,
and $\psi$ is a first-order sentence.
Further, let $\A$ be a first-order structure over the
vocabulary of $\psi$.
We now show how to generate a planning problem $P$ 
that has solution iff $\A\models\Phi$.

Following PMB, $P$ has atoms of the form Holds-$\FT[\theta]$
for every subformula $\theta$ of $\psi$ and with parameters
that match the free variables in $\theta$. For example,
if $\Phi$ is the formula for SAT:
\begin{alignat*}{1}
\Phi_\SAT &= (\exists T^\text{Var})\psi_\SAT \,, \\
\psi_\SAT &= (\forall y\in \text{Cls})(\exists x\in\text{Var}) \notag \\
          &\quad\quad\quad\quad
               [(P(x,y) \land T(x)) \lor (N(x,y) \lor \neg T(x))]
\end{alignat*}
where `Var' and `Cls' are the types for variables and clauses,
then there is an atom Holds-$\FT[\theta]$ with parameters
$\tup{x,y}$ for the subformula $\theta(x,y)=N(x,y)\lor\neg T(x)$.
In $\psi_\SAT$, the relation $P(x,y)$ (resp.\ $N(x,y)$) denotes
that the variable $x$ appears positively (resp.\ negatively) in
the clause $y$; these relations are fixed for a given SAT instance
and their interpretation is given in the structure $\A$ that
encodes the instance.
On the other hand, the existentially-quantified relation $T$
encodes the sought model in a way that $T(x)$ is true iff the
variable $x$ is assigned the truth-value true.

The problem $P$ has actions for choosing the valuation $T$,
that is represented by fluents of the form `$\texttt{T}(?x)$'
and `$\texttt{not\_T}(?x)$', and actions for building a
proof for $\psi_\SAT$ that are designed to work by following
the recursive structure of $\psi_\SAT$.
The goal of the problem is defined as the single fluent
Holds-$\FT[\psi_\SAT]$ which has no parameters as $\psi_\SAT$
is assumed to be a sentence.

\smallskip

In our approach, we make use of the same fluents that denote
the validity of subformulas and the quantified relations.
The difference though is in how the quantified relations
are built and \emph{interleaved} with the proofs of the
subformulas. 
The interleaving issue does not arise for \SOE\ as there is
just one interpretation to construct for each existentially-quantified
relation, but for general formulas, one needs to construct
and test many different interpretations.
To make this point clear, let us consider the formula
for UNSAT that involves a universally-quantified unary relation $T$:
\begin{alignat*}{1}
\Phi_\UNSAT &= (\forall T^\text{Var})\psi_\UNSAT \\
\psi_\UNSAT &= (\exists y\in\text{Cls})(\forall x\in\text{Var}) \\
            &\quad\quad\ [(\neg P(x,y) \lor \neg T(x))\land(\neg N(x,y)\lor T(x))] \,.
\end{alignat*}
This formula
says that for every relation $T$ over variables (that encodes
a model), there is a clause $y$ such that for every variable $x$,
if $x$ appears positive in $y$, then $x$ is false, and if $x$
appears negative in $y$, $x$ is true.
UNSAT can thus be automatically translated into \STRIPS by 
considering actions that ``iterate'' over all possible relations
$T$, and actions for obtaining the fluent
Holds-$\FT[\psi_\UNSAT]$ for each such $T$.

For unary $T$, there are $2^n$ different relations on $n$
variables. Each relation can be encoded with a binary string
of length $n$ (one bit per variable) so that the $i$th bit
is 1 iff the $i$th variable $x_i$ belongs to $T$.
Thus, to iterate over all relations is equivalent to iterating over
all such strings. This can be done by starting with the empty
relation, corresponding to `0\ldots00', and successively
``adding 1'' until reaching `1\ldots11'.
Figure~\ref{fig:unsat} shows 5 actions that do the iteration;
these actions make use of the extra fluents Need- and
Holds-$\FT[\Phi_\UNSAT]$, for the second-order formula
$\Phi_\UNSAT$, and the fluents `Marker($x$)',
`VarFirst($x$)', `VarSucc($x,y$)' and `VarLast($x$)'.
The last three type of fluents are \emph{static} fluents
that are set in the initial situation and implement an
static order of the objects that refer to propositional
variables.

Let us briefly explain how the iteration works.
The initial situation contains
the static fluents defining the Var and Cls types,
and the binary relations $N(x,y)$ and $P(x,y)$, plus
the fluent Need-$\FT[\Phi_\UNSAT]$, while the goal
contains only Holds-$\FT[\Phi_\UNSAT]$.
Initially, the only applicable action is A1 that
sets $T$ as the empty relation and adds 
Need-$\FT[\psi_\UNSAT]$.
Although not shown, this fluent triggers actions
for achieving a proof of $\psi_\UNSAT$ for the
current $T$, which as a side effect add 
Holds-$\FT[\psi_\UNSAT]$ once $\psi_\UNSAT$
is proved.
If $\psi_\UNSAT$ is proved, A2 becomes the only operator
applicable and it removes
Holds-$\FT[\psi_\UNSAT]$ and adds Marker($x$) for the
first variable $x$ in the static order.
After A2, A3 must be applied and it
changes $T$
from the model corresponding to 0\ldots00 to the
model corresponding to 0\ldots01, and adds 
Need-$\FT[\psi_\UNSAT]$ that asks for a proof
of $\psi_\UNSAT$ for the new $T$.
This iteration proceeds in a similar manner until
proving $\psi_\UNSAT$ for the last model corresponding
to 1\ldots11, a time at which A5 is the only applicable
and adds Holds-$\FT[\Phi_\UNSAT]$.\footnote{It
should be noted that the fluents Marker, etc.\
are for the SO quantifier $T$. Later, when composing
translations, each SO quantifier is associated with
its own fluents.}

\begin{figure}[t]
\centering
\fbox{
  %\resizebox{3.1in}{!}{
  \resizebox{!}{1.5in}{
    \begin{minipage}{3.6in}
    \begin{tabbing}
    Pre: \= \kill
    [A1] \textsc{StartProof}: \\[.2em]
    Pre: \> Need-$\FT[\Phi_\UNSAT]$ \\
    Eff: \> for each $x\in\text{Var}$: not-T($x$) \,, \\
         \> $\neg$Need-$\FT[\Phi_\UNSAT]$ \,,\, Need-$\FT[\psi_\UNSAT]$ \\[1em]
    %%%
    [A2] \textsc{NextIterate}($x$): \\[.2em]
    Pre: \> Holds-$\FT[\psi_\UNSAT]$ \,,\, VarFirst($x$) \\
    Eff: \> $\neg$Holds-$\FT[\psi_\UNSAT]$ \,,\, Marker($x$) \\[1em]
    %%%
    [A3] \textsc{NextOverExcluded}($x$): \\[.2em]
    Pre: \> Marker($x$) \,,\, not-T($x$) \\
    Eff: \> $\neg$not-T($x$) \,,\, T($x$) \,,\, $\neg$Marker($x$) \,,\, Need-$\FT[\psi_\UNSAT]$ \\[1em]
    %%%
    [A4] \textsc{NextOverIncluded}($x,y$): \\[.2em]
    Pre: \> Marker($x$) \,,\, T($x$) \,,\, VarSucc($x,y$) \\
    Eff: \> $\neg$T($x$) \,,\, not-T($x$) \,,\, $\neg$Marker($x$) \,,\, Marker($y$) \\[1em]
    %%%
    [A5] \textsc{FinishProof}($x$): \\[.2em]
    Pre: Marker($x$) \,,\, T($x$) \,,\, VarLast($x$) \\
    Eff: $\neg$T($x$) \,,\, not-T($x$) \,,\, $\neg$Marker($x$) \,,\, Holds-$\FT[\Phi_\UNSAT]$
    \end{tabbing}
    \end{minipage}
  }
}
\caption{Actions for iterating over the $2^n$ unary relations $T$
  that encode the $2^n$ truth-assignments for $n$ propositional
  variables in UNSAT.}
\label{fig:unsat}
\end{figure}

It should be noted that this iteration method also  works
for relations of arity $k>1$.
The only change needed is to replace $x$ and $y$ for 
$k$-tuples of variables.
Likewise, the fluents that implement the static order
on variables are changed by fluents that  implement an
static order over $k$-tuples.

In the case of SAT, our translation differs from PMB's by
having fluents for the SO formula $\Phi_\SAT$ and by ``passing control''
to the actions
that prove Holds-$\FT[\psi_\SAT]$ in a way similar to UNSAT;
Fig.~\ref{fig:sat} shows the operators for SAT.

The translations for SAT and UNSAT share a common protocol
that controls which operators become active or inactive
depending on which part of the formula is being proved.
This protocol is designed to allow the composition of
the two types of SO quantifiers, and thus to make a
translation for \PH.
For example, a formula in $\Sigma^p_2$ of the form
$\Phi=(\exists T^{t_1})(\forall R^{t_2})\psi$ can
be decomposed as
\begin{equation}
\Phi = (\exists T^{t_1})\Psi \quad\text{with}\quad \Psi = (\forall R^{t_2})\psi \,, 
\end{equation}
so that operators like the ones for SAT, with fluents
Need- and Holds-$\FT[\Psi]$, may be combined with operators
like the ones for UNSAT with fluents Need- and Holds-$\FT[\psi]$.
In such combination, the E5 operator for the inner SO
existential must delete the fluents for the subformulas in
order to have a ``fresh state'' for the next proof.

\begin{figure}[t]
\centering
\fbox{
  \resizebox{3.1in}{!}{
    \begin{minipage}{3.6in}
    \begin{tabbing}
    Pre: \= \kill
    [E1] \textsc{StartProof}: \\[.2em]
    Pre: \> Need-$\FT[\Phi_\SAT]$ \\
    Eff: \> for each $x\in\text{Var}$: not-T($x$) \,,\, $\neg$Need-$\FT[\Phi_\SAT]$ \,,\, Guess \\[1em]
    %%%
    [E2] \textsc{SetTrue}($x$): \\[.2em]
    Pre: \> Guess \quad;\quad
    Eff: T($x$) \,,\, $\neg$not-T($x$) \\[1em]
    %%%
    [E3] \textsc{SetFalse}($x$): \\[.2em]
    Pre: \> Guess \quad;\quad
    Eff: not-T($x$) \,,\, $\neg$T($x$) \\[1em]
    %%%
    [E4] \textsc{ProofSubformula}: \\[.2em]
    Pre: \> Guess \quad;\quad
    Eff: $\neg$Guess \,,\, Need-$\FT[\psi_\SAT]$ \\[1em]
    %%%
    [E5] \textsc{FinishProof}: \\
    Pre: \> Holds-$\FT[\psi_\SAT]$ \quad;\quad
    Eff: $\neg$Holds-$\FT[\psi_\SAT]$ \,,\, Holds-$\FT[\Phi_\SAT]$
    \end{tabbing}
    \end{minipage}
  }
}
\caption{Actions that implement the second-order existential
  quantifier in SAT.}
\label{fig:sat}
\end{figure}
 
\subsection{Horizon Windows for SAT-based Planners}

Given a SO formula $\Phi$ describing property $\Pi$ and a
structure $\A$ encoding an instance of the problem, one
wants to decide whether the instance satisfy the property
or not; i.e., $\A\models\Phi$. The tool just described 
generates in polytime (in the size $\|\A\|$) an \STRIPS
problem $P$ that has solution iff $\A\models\Phi$.
Thus, $P$ can be solved with any complete planner to
answer the original question. However, SAT-based planners
are inherently incomplete when there is no solution as
they continue the search forever unless an upper bound
on the length of the plan is given in advance.\footnote{Of
course, $2^n$ where $n$ is the number of fluents in $P$
is a trivial upper bound. Clearly, for $n\geq100$ (or less)
such a bound is inconsequential. Very often problems have
more than 100 fluents.}

PMB shows how to calculate \emph{tight} lower and upper bounds
$l(\psi)$ and $u(\psi)$ on the number of actions needed to prove
a first-order sentence $\psi$ once the interpretations of the
relational symbols occurring in $\psi$ are fixed.
We use such estimates to calculate lower and upper bounds on
the length of \emph{parallel plans} for SO formulas.
We proceed inductively
in the structure of $\Phi$, for $X\in\{l,u\}$:
\begin{enumerate}[1.]
\item if $\Phi=\psi$ (first-order formula), then $X(\Phi)=X(\psi)$.
\item If $\Phi=(\exists R^t)\Psi$, then $X(\Phi)=3+[\![X=u]\!]+X(\Psi)$.
\item If $\Phi=(\forall R^t)\Psi$, then
  $X(\Phi)=2^{\#t}[X(\Psi)+3]-1$%   2^{\#t+1}+2^{\#t}X(\Psi)=2^{\#t}[2+X(\Psi)]$
where
  $\#t$ is number of $t$-type tuples, $2^{\#t}$
  is number of relations $R$, and $2^{\#t+1}-2$ is number
  of bits flipped when incrementing a $\#t$-bit counter
  from zero to its maximum \cite{cormen:algorithms}.
\end{enumerate}
In 2, $[\![X=u]\!]$ is 1 or 0 whether $X$ refers to the 
upper bound $u$ or not. It is not hard to see that one
can get the unique expression $X(\Phi)=4+X(\Psi)$ for the
lower and upper bounds in 2 by pushing the lower bound 1 unit.
