% Chapter 1

\chapter*{Introducción} % Write in your own chapter title
\label{Intro}
\lhead{\emph{Introducción}} % Write in your own chapter title to set the page header
\addcontentsline{toc}{chapter}{Introducción}
% Descripción general
En Ciencias de la Computación se entiende por \textbf{problemas de decisión}
los problemas matemáticos en los que se plantea una pregunta sobre una entrada
arbitraria, cuya respuesta puede ser afirmativa o negativa.
Los problemas de decisión son interesantes porque permiten modelar y razonar
sobre aplicaciones reales. Por ejemplo, el problema de decisión de
$k$-colorabilidad pregunta si los vértices de un grafo son coloreables 
con $k$ colores distintos, de modo de que a ningún par de vértices se les
asigne el mismo color. En un compilador, la asignación de valores
frecuentemente utilizados a registros para
mejorar la eficiencia del código generado puede modelarse como
$k$-colorabilidad: los vértices del grafo son las variables, y si dos variables
se necesitan al mismo tiempo se coloca un arco entre ellas. Al encontrar una
forma de colorear este grafo con $k$ colores se ha hallado una forma de
utilizar $k$ registros distintos para almacenar las variables.

Los problemas de decisión pueden agruparse en distintas clases según su
complejidad. Una importante clase de complejidad es \textbf{NP} (siglas de
\textit{Nondeterministic Polynomial time}), a la cual pertenecen problemas
interesantes como $k$-colorabilidad, satisfacibilidad proposicional
(\textbf{SAT}) y hallar un camino \textit{hamiltoniano} en un grafo.

El área de investigación de planificación automática estudia una forma general
de resolución de problemas utilizando \textbf{planes}. En un problema de planificación
se especifica un estado inicial, un estado final y un conjunto de acciones posibles,
y el \textbf{planificador} construye una secuencia de acciones que conduce
desde el estado inicial hasta el estado final, resolviendo el problema.

% Concretar

Este trabajo se enfoca en el desarrollo de una herramienta que permite
traducir, o como suele decirse formalmente, \textbf{reducir} un problema de
decisión perteneciente a la clase \textbf{NP}, codificado en un lenguaje
basado en la lógica de segundo orden, a un problema de
planificación automática. La motivación para realizar esta herramienta es que
existen planificadores muy eficientes, pero no es trivial
expresar un problema de decisión cualquiera en términos de estado inicial,
final y acciones, como lo requieren los
planificadores. A veces, la utilización de un lenguaje declarativo es una forma
más concreta y directa para la descripción de un problema de decisión en
particular. En estos casos, es preferible modelarlo en un lenguaje de más alto
nivel y que la herramienta lo reduzca a un formato de entrada aceptado por
los planificadores, los cuales pueden proceder entonces a computar su solución.

% Trabajos anteriores
Otros autores han utilizado la lógica de segundo orden como un lenguaje de
programación declarativo anteriormente. \cite{cadoli:npspec}.

% Objetivos generales y específicos

% Organización

1. RESULTADOS ESPECÍFICOS A LOS QUE SE QUIERE LLEGAR 
(EL PRODUCTO QUE SE DESEA OBTENER) Y SU POSIBILIDAD DE APLICACIÓN: 

Resultados:
1. Una herramienta que procese la descripción lógica de un problema perteneciente a una clase de complejidad particular y que genere automáticamente un problema de planificación equivalente formulado en el lenguaje PDDL (Planning Domain Definition Language). 
2. Una interfaz para este sistema que permita un acceso flexible y rápido al traductor.
Posibilidad de Aplicación:
Gracias al sistema propuesto, problemas cuya aplicación es ampliamente conocida
en el área de computación, tales como hallar un clique de tamaño k o encontrar un camino hamiltoniano en un grafo, podrán ser expresados en un lenguaje conciso y ser traducidos en archivos de entrada para un planificador sin la intervención del usuario.

2. ACTIVIDADES QUE INVOLUCRA EL PROYECTO: 
a. Investigación sobre el material teórico que permiten relacionar la teoría de complejidad computacional con complejidad descriptiva (lógica)
b. Especificación del lenguaje lógico propuesto para realizar formulaciones, gramática y semántica de las construcciones
c. Definición de la traducción del lenguaje lógico a PDDL, demostración de la correctitud y completitud de esta equivalencia
d. Desarrollo del analizador léxico y sintáctico que permita reconocer este lenguaje
e. Implementación de algoritmos para el post-procesamiento del árbol sintáctico para la simplificación de la fórmula lógica a formas canónicas
f. Implementación de la traducción definida anteriormente sobre el árbol sintáctico de la fórmula procesada
g. Evaluación experimental del desempeño de planificadores (del estado del arte) sobre fórmulas que corresponden a varios dominios
h. Programación de una herramienta integral web que sirva como una interfaz ante el usuario para acceder al sistema de traducción de forma cómoda

3. PUNTOS DE INTERÉS QUE HAN DE SER TRATADOS DURANTE LA EJECUCIÓN DEL PROYECTO:
a. Teoría de complejidad descriptiva
b. Equivalencia entre clases de complejidad y fragmentos del lenguaje PDDL
c. Traducciones válidas entre fórmulas lógicas y dominios de planificación, optimización de las traducciones para su resolución con planificadores actuales

\cite{russell}
